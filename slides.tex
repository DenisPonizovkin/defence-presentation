%\documentclass[10pt,pdf,hyperref={unicode},fleqn]{beamer
\documentclass[10pt,xcolor={usenames,dvipsnames}]{beamer}

\usepackage[T2A]{fontenc}       %поддержка кириллицы
\usepackage[utf8]{inputenc}
\usepackage{longtable} 
\usepackage{cancel}
\usepackage{amsmath}
\usepackage{tikz}
\usepackage{graphicx}
\usepackage{cancel}
\usetheme{Darmstadt}
\setbeamercolor{block title}{bg=white!30,fg=black}
\newenvironment{variableblock}[3]{%
  \setbeamercolor{block body}{#2}
  \setbeamercolor{block title}{#3}
  \begin{block}{#1}}{\end{block}}

%% \usefonttheme{structurebold}
\usefonttheme[onlymath]{serif}
\setbeamertemplate{footline}{\hspace*{.5cm}\scriptsize{
\hspace*{50pt} \hfill\insertframenumber/\inserttotalframenumber\hspace*{.5cm}}} 
%\setbeamercovered{transparent}
%\setbeamercolor{color1}{bg=white!90!black,fg=white}
\setbeamercolor{normal text}{bg=white,fg=black}
\setbeamercolor{frametitle}{fg=black,bg=gray}
%\insertframenumber
%% \setbeamercolor{footline}{fg=blue}
%% \setbeamerfont{footline}{series=\bfseries}
%\setbeamerfont{page number in head/foot}{size=\small}
%\hfill\insertframenumber/\inserttotalframenumber
%\setbeamerfont{page number in head/foot}{size=\large}
\theoremstyle{break}
\newtheorem{assert}{Утверждение}[section]
\newtheorem{assertORS}{Утверждение ОРС}
\newtheorem{assertSRS}{Утверждение СРС}
\newtheorem{myproof}{Доказательство}[section]

%% \defbeamertemplate*{footline}{shadow theme}
%% {%
%%   \leavevmode%
%%   \hbox{\begin{beamercolorbox}[wd=.5\paperwidth,ht=2.5ex,dp=1.125ex,leftskip=.3cm plus1fil,rightskip=.3cm]{author in head/foot}%
%%     \usebeamerfont{author in head/foot}\insertframenumber\,/\,\inserttotalframenumber\hfill\insertshortauthor
%%   \end{beamercolorbox}%
%%   \begin{beamercolorbox}[wd=.5\paperwidth,ht=2.5ex,dp=1.125ex,leftskip=.3cm,rightskip=.3cm plus1fil]{title in head/foot}%
%%     \usebeamerfont{title in head/foot}\insertshorttitle%
%%   \end{beamercolorbox}}%
%%   \vskip0pt%
%% }
\def\top{\text{{\it top-N}}}
\def\p{\text{p}}

\newcommand{\ru}{\mathcal{R}_i}
\newcommand{\rt}{\mathcal{R}_j}
\newcommand{\R}{\mathcal{R}}
\newcommand{\rin}{\overline{\rho_0}}


\begin{document}

%1
\title{Модель рекомендательной системы на нечетких множествах как эффективное расширение коллаборативной модели}
\author{Понизовкин Денис Михайлович}
\institute{ИПС им. А. К. Айламазяна РАН \\
    \vspace{0.7cm}
    Научный руководитель:  к. т. н. Амелькин Сергей Анатольевич \\
    \vspace{0.7cm}
}
\date{
	Специальность: 05.13.17
}
\frame{\titlepage}

%SLIDE: 1
\begin{frame}
\scriptsize{
\frametitle{Исходные данные}
\begin{columns}[T]
    \column{.5\textwidth} % Left column and width
		\begin{variableblock}{Пользователи}{ }{bg=green!9,fg=blue}
			\begin{itemize}
				\item $u \in U \subset \mathbb{N}$ --- идентификаторы пользователей РС;
				\item $X$ --- множество наименований характеристик пользователей;
				\item $c_X : U \rightarrow X$, $c_X(u)$ --- структура данных,
					представляющую информацию о пользователе.
				%\item $w_U: U \times X \rightarrow[0,1]$ --- значение
				%	характеристики;
			\end{itemize}
		\end{variableblock}

  \column{.5\textwidth} % Left column and width
		\begin{variableblock}{Объекты}{ }{bg=green!9,fg=blue}
			\begin{itemize}
			\item $i \in I \subset \mathbb{N}$ --- идентификаторы объектов
				предметной области РС;
			\item $Y$ --- множество наименований характеристик объектов;
			\item $c_Y : I \rightarrow Y$, $c_Y(i)$ --- структура данных,
				представляющую информацию об объекте.
			%\item $w_I: I \times Y \rightarrow[0,1]$ --- значение
			%	характеристики;
			\end{itemize}
	\end{variableblock}
\end{columns}

    \begin{variableblock}{Расстояние между пользователем и объектом}{ }{bg=green!9,fg=blue}
		$\rho: U \times I \rightarrow [0,1]$ --- функция расстояния между
		пользователем и объектом.
				\begin{equation*}
				P =
					\begin{pmatrix}
						\rho(1,1)&  ... &  \bot  & ... & \rho(1,n)  \\
						\rho(2,1)&   \bot  & ... & ... & \rho(2,n)  \\
						\bot        &  .. &  \bot  & ... & \bot  \\
						\rho(m,1)&  ... & ... &  \bot  & \rho(m,n)  \\
					\end{pmatrix}
				\end{equation*}

		$\bot$ --- неизвестное значение \\

		\begin{variableblock}{Расстояние между пользователем и объектом}{ }{bg=green!9,fg=blue}
    \end{variableblock}

\begin{variableblock}{Расстояние между пользователем и объектом}{ }{bg=green!9,fg=blue}
$u \R i \Leftrightarrow \rho(u, i) \le \varepsilon_0$ --- отношение близости
$\varepsilon_0 \in \varepsilon(0)$.
\end{variableblock}
	}
\end{frame}


\begin{frame}
	\frametitle{Подзадачи РС}
	\scriptsize{
	\begin{columns}[T]
		\column{.5\textwidth} % Left column and width
		\begin{variableblock}{$p$ --- прогнозирование}{ }{bg=green!9,fg=blue}
		\begin{itemize}
			\item Необходимо спрогнозировать неизвестное значение
			$\rho(u_a,i_p)$ путем алгоритмического вычисления значения прогнозной
			функции $\overline{\rho}(u_a,i_p): U \times I \rightarrow [0,1]$,
			где $i_p$ --- прогнозируемый объект;

			\item Задача решена качественно, если
			$|\rho(u_a, i_p) - \overline{\rho}(u_a,i_p)| \le \varepsilon_0$;

		\end{itemize}
		\end{variableblock}

		\begin{variableblock}{Обучающее множество}{ }{bg=green!9,fg=blue}
		$P_0 \subset P = \{\rho(u_a, i_0)\}$ --- обучающее множество,
		$\rho(u_a, i_0) \ne \bot$.
		\end{variableblock}


		\column{.5\textwidth} % Left column and width
		\begin{variableblock}{$\top$}{ }{bg=green!9,fg=blue}
		\begin{itemize}
			\item формирование подмножества объектов $I_{\top} = \{i:(u_a \R i) \wedge
			\rho(u_a,i) = \bot\} \wedge |I_{\top}|=N$. Так как неизвестно,
			выполняется ли отношение $u_a \R i$ в силу того, что $\rho(u_a,i) =
			\bot$, то выполнение отношения $u_a \R i$ определяется по значению
			прогнозной функции $\overline{\rho}(u_a,i_p)$.

		\end{itemize}
		\end{variableblock}

		\begin{variableblock}{Обучающее множество}{ }{bg=green!9,fg=blue}
		$P_0 \subset P = \{\rho(u_a, i_0)\}$ --- обучающее множество,
		$\rho(u_a, i_0) \le \epsilon_0$.\\
		\end{variableblock}

	\end{columns}

	}
\end{frame}


\begin{frame}
	\frametitle{Определение модели РС}
	\scriptsize{
		\begin{variableblock}{Модель РС}{ }{bg=green!9,fg=blue}
		\begin{equation}
		(c_X; c_Y; \Pi; \mathcal{E}_{t})
		\end{equation}
		\end{variableblock}

	\begin{itemize}
		\item $\Pi$ --- правило алгоритмического вычисления $\overline{\rho}(u,i)$;
		\item $\mathcal{E}_{t}: \overline{P}_{\bot} \times
		P_{\bot} \rightarrow [0,1]$ --- функция оценки
			качества решения задачи $t \in \{p, \top\}$. Если
			$\mathcal{E}_{t} \le \epsilon_0$, задача решена качественно;
		\item $P_{\bot}$ --- тестовое множество;
		\item $\overline{P}_{\bot}$ --- результирующее множество.
	\end{itemize}
	}
\end{frame}


\begin{frame}
	\frametitle{Модели коллаборативных РС}
	\scriptsize{
		\begin{columns}[T]
		\column{.5\textwidth} % Left column and width
		\begin{variableblock}{Объектно-ориентированная (далее $OOM$)}{ }{bg=green!9,fg=blue}
		\begin{itemize}
		\item Фундаментальное эвристическое утверждение:
			\\$(u_a \R i)$ $\wedge$ $(i \rt j) \Rightarrow (u_a \R j)$;
		\item Отношение близости объектов: $(i \rt j) \Leftrightarrow \delta_i(i,j) > \Delta \in [0,1)$;
		\item Мера сходства объектов: $\delta_i: I \times I \rightarrow [0,1]$;
		\item $\Pi = \Pi_{OOM}$: $(i \rt i_0) \Rightarrow \overline{\rho}(u_a,
			i) \le \epsilon_0$;
		\item Решение задачи $\top$: $I_{\top} = \{i: i \rt I_0\}, I_0 = \{i_0:
			\rho(u_a, i_0) \in P_0\}$;
		\item Используется компанией Amazon.
		\item $\mathcal{E}_{\top} \in \{P, AveP, NDCG, ...\}$
		\end{itemize}
		\end{variableblock}

		\column{.5\textwidth} % Left column and width
		\begin{variableblock}{Субъектно-ориентированная (далее $COM$)}{ }{bg=green!9,fg=blue}
		\begin{itemize}
		\item Фундаментальное эвристическое утверждение:
			\\$(u_a \ru u) \text{ для } P_0 \Rightarrow (u_a \ru u) \text{ для }
				P_{\bot}$;
		\item Отношение близости объектов: $(u \ru v) \Leftrightarrow \delta_u(u,v) > \Delta \in [0,1)$;
		\item Мера сходства объектов: $\delta_u: U \times U \rightarrow [0,1]$;
		\item $\Pi = \Pi_{COM}$: $u_a \ru u \Rightarrow
			|\overline{\rho}(u_a, i_p) - \rho(u, i_p)| \le \epsilon_0$;
		\item решение задачи прогнозирования:
			$\overline{\rho}(u_a, i_p) = f(\{\rho(u, i_p)\})$, $u_a \ru u$,
				$\rho(u, i_p) \ne \bot$;
		\item $c_X(u) = \{\rho(u, i)\}$;
		\item $X = I$;
		\item $w_U(u, x) = \rho(u, x)$;
		\item Используется компанией Netflix.
		\item $\mathcal{E}_{p} \in \{MAE, NMAE, RMSE, ... \}$
		\end{itemize}
		\end{variableblock}
		\end{columns}
	}
\end{frame}

\begin{frame}
	\frametitle{Эффективность коллаборативных моделей по критерию качества
	решения}

	\begin{variableblock}{}{ }{bg=green!9,fg=blue}
	Модель эффективна по некоторому критерию, если ее правила вывода
	удовлетворяют этому критерию независимо от дополнительных ограничений
	или условий.
	\end{variableblock}

	\begin{variableblock}{Эффективность по критерию качества}{ }{bg=green!9,fg=blue}
		$\mathcal{E}_t(P_{\bot}, \overline{P}_{\bot}) \le \epsilon_0$
	\end{variableblock}

	\begin{columns}[T]
	\column{.5\textwidth} % Left column and width
	\begin{variableblock}{Достаточное условие эффективности $OOM$}{ }{bg=green!9,fg=blue}
		Транзитивность отношения близости:\\
		$(i \rt j) \wedge (i \rt k) \Rightarrow (j \rt k), i, j, k \in I_0
		\bigcup I_{\top} \bigcup I_{\bot}, I_{\bot} = \{i_{\bot}: \rho(u_a,
		i_{\bot}) \in P_{\bot}\}, I_0=\{i_0: \rho(u_a, i_0) \in P_0\}$.
	\end{variableblock}

	\column{.5\textwidth} % Left column and width
	\begin{variableblock}{Достаточное условие эффективности $COM$}{ }{bg=green!9,fg=blue}
		Транзитивность отношения близости:\\
		$(u_a \ru u) \wedge (u_a \ru v) \Rightarrow (u \ru v)$.
	\end{variableblock}
	\end{columns}

	\begin{variableblock}{}{ }{bg=green!9,fg=blue}
		Выполнение достаточных условий зависит выбора разработчиками $\delta_i, \delta_u$.
	\end{variableblock}

\end{frame}

\begin{frame}
	\frametitle{Эффективность коллаборативных моделей по критерию
	вычислительной сложности алгоритмов}

	\begin{columns}[T]
	\column{.5\textwidth} % Left column and width
	\begin{variableblock}{$OOM$, задача $\top$}{ }{bg=green!9,fg=blue}
		Асимптотическая сложность решения $O(C \cdot |I|)$.
	\end{variableblock}

	\column{.5\textwidth} % Left column and width
	\begin{variableblock}{$COM$, задача $p$}{ }{bg=green!9,fg=blue}
	Асимптотическая сложность решения $O(|U|)$.
	\end{variableblock}
	\end{columns}

	\begin{variableblock}{Эффективность коллаборативных моделей по критерию
	вычислительной сложности алгоритмов}{ }{bg=green!9,fg=blue}
	With millions of users and items, existing CF-based recommender
		systems suffer serious scalability problems.
		\footnote{
			G. Karypis. Evaluation of item-based top-N recommendation
			algorithms //
			Proceedings of the International Conference on Information and
			Knowledge Management, 2001, pp. 247-254.
		}
	\end{variableblock}
\end{frame}

\begin{frame}
	\frametitle{Эффективность коллаборативных моделей по критерию стабильности}

	\scriptsize{
	\begin{variableblock}{}{ }{bg=green!9,fg=blue}
		\scriptsize{
		Стабильность --- свойство системы генерировать качественное решения
		независимо от свойств исходных данных.
		}
	\end{variableblock}

	\begin{columns}[T]
	\column{.4\textwidth} % Left column and width
%The abundance of information available on the Web and in Digital Libraries, in
%combination with their dynamic and heterogeneous nature, has determined a rapidly
%increasing difficulty in finding what we want when we need it and in a manner which
%best meets our requirements.
		\begin{variableblock}{Динамика исходных данных}{ }{bg=green!9,fg=blue}
		\scriptsize{
			$(u_a \ru u) \text{ для } T_0 \cancel{\Rightarrow} (u_a \ru u) \text{ для
		} T^a_{\times}$; \\
\noindent\rule{4cm}{0.4pt} \\
		\tiny{
		Koren. Collaborative filtering with temporal dynamics // Proceedings of the
		15th ACM SIGKDD international conference on Knowledge discovery and data
		mining, 2009, p. 447–456}}.
	\end{variableblock}

	\column{.6\textwidth} % Left column and width
	\begin{variableblock}{Гетерогенность предпочтений пользователя}{ }{bg=green!9,fg=blue}
		\scriptsize{
		$\Big((u_a \R  i)\wedge (u_a \R j) \Rightarrow (i \rt j)\Big) = b_1 \in
		\{0, 1\} \Rightarrow $\\
		$\Big((u_a \R i) \wedge (i \rt j) \Rightarrow (u_a \R j) \Big) = b_2 \in
		\{0, 1\}$ --- фундаментальное утверждение $OOM$.
		}
		\noindent\rule{4cm}{0.4pt} \\
				\tiny{
				Koren. Collaborative filtering with temporal dynamics // Proceedings of the
				15th ACM SIGKDD international conference on Knowledge discovery and data
				mining, 2009, p. 447–456}.
	\end{variableblock}
	\end{columns}
	\begin{variableblock}{Эффективность коллаборативных моделей по критерию стабильности}{ }
		{bg=green!9,fg=blue}
		\scriptsize{
		Если исходные данные обладают свойством динамики или гетерогенности,
		эвристические утверждения не выполняются, поэтому в общем случае
		коллаборативные модели неэффективны по заданному критерию.
		}
	\end{variableblock}
	}
\end{frame}

\begin{frame}
	\frametitle{Проблемы использования оценок качества}

	\begin{variableblock}{}{ }{bg=green!9,fg=blue}
		\scriptsize{
		The challenge of selecting an appropriate metric is compounded by the large
		diversity of published metrics that have been used to quantitatively evaluate
		the accuracy of recommender systems. This lack of standardization is damaging
		to the progress of knowledge related to collaborative filtering recommender
		systems. With no standardized metrics within the field, researchers have continued
		to introduce new metrics when they evaluate their systems.}\\
		\noindent\rule{4cm}{0.4pt} \\
				\tiny{
				J. L. Herlocker. Evaluating collaborative filtering recommender
				systems. /
				 J. L. Herlocker, J. A. Konstan, L. G. Terveen, J. Riedl //
				 ACM Trans. Inf. Syst Т 22(1), pp 5-53, 2004
				}.
	\end{variableblock}

	\begin{variableblock}{Оценка $\mathcal{E}_{\top}$}{ }{bg=green!9,fg=blue}
		Оценка основана на эвристическом утверждении $OOM$:\\
		$\mathcal{E}_{\top} = g(s(i)), s(i) = 1,$ если $\exists i_{\bot}: i \rt
		i_{\bot}$ \\
	\end{variableblock}

	\begin{variableblock}{Объективность оценки $\mathcal{E}_{\top}$}{ }{bg=green!9,fg=blue}
		\scriptsize{
		Если исходные данные обладают свойством динамики или гетерогенности,
		эвристические утверждения не выполняются, поэтому в общем случае
		значение $\mathcal{E}_{\top}$ не является объективным показателем
		качества.
		}
	\end{variableblock}
\end{frame}

\begin{frame}
	\frametitle{Цели исследования}
	\begin{enumerate}
	\item Разработать математическую модель РС, эффективно расширяющую
		коллаборатвиную модель по критериям качества решения
		и вычислительной сложности;
	\item Разработать программное обеспечение для коллаборативной и
		введенной в исследовании моделей, на базе которого получить
		практические результаты.
	\end{enumerate}
\end{frame}

%\begin{frame}
%	\frametitle{Новизна и актуальность}
%	%\scriptsize{
%	\begin{columns}[T]
%	\column{.5\textwidth} % Left column and width
%	\begin{variableblock}{Актуальность}{ }{bg=green!9,fg=blue}
%		Определяется:
%		\begin{enumerate}
%		\scriptsize{
%		\item существованием открытых проблем, решение которых повысит
%			эффективность существующих РС и будет способствовать формированию
%			новых эффективных РС.
%			}
%		\end{enumerate}
%	Подтверждается:
%	\begin{enumerate}
%	\scriptsize{
%	\item академическим спросом: ежегодно проводятся международные конференции
%		RecSys, посвященные РС;
%	\item коммерческий спрос: РС все чаще встраиваются в веб-сервисы и
%		существуют компании (Retail Rocket), предлагающие движки РС, которые
%			могут встроены в любой интрнет магазин.
%		}
%	\end{enumerate}
%	\end{variableblock}
%
%	\column{.5\textwidth} % Left column and width
%	\begin{variableblock}{Научная новизна}{ }{bg=green!9,fg=blue}
%		\begin{enumerate}
%			\scriptsize{
%			\item Разработано новое представление данных РС на основе
%				неиспользуемой ранее в РС теории нечетких множеств;
%			\item Разработаны новые алгоритмы решения подздач и новый алгоритм
%			решения подзадачи прогнозирования при использовании коллаборативных
%			правил вывода, являющийся расширением существующего и гарантирущим
%				эффективность решения;
%			\item Разработана новая оценка эффективности, коррелирующая с
%				существующими и являщийся их обощением.
%				}
%		\end{enumerate}
%	\end{variableblock}
%\end{columns}
%%}
%\end{frame}


%%%%%%%%%%%%%%%%%%%%%%%%%%%%%
% chapter: fuzzy
%%%%%%%%%%%%%%%%%%%%%%%%%%%%%%
\begin{frame}
	\frametitle{Нечеткая модель. Представление контентов.}
  \begin{itemize}
	  \item $c_X(u) = \{(x_i | \mu_u(x_i )) \}, \mu_u(x_i) \in [0,1], x
		  \subseteq X$ --- {\it контент пользователя}
	  \item $c_Y(i) = \{(y_k | \nu_i(y_k )) \}, \nu_i(y_k) \in [0,1], y
		  \subseteq Y$ --- {\it контент объекта}
	  \item $c_X(u) = \varnothing$, если $\mu_u(x_k) \equiv 0$---
		  {\it пустой контент пользователя}
	  \item $c_Y(i) = \varnothing$, если $\nu_i(y_k) \equiv 0$ ---
		  {\it пустой контент объекта}
	  \item $c_X(u) \bigcap c_X(u^{\prime}) = c_X(u^{\prime \prime}):
		  \forall x \in X$ $\mu_u(x) =
	  \min(\mu_{u^{\prime}}(x), \mu_{ u^{\prime \prime}}(x))$ --- {\it пересечение контентов пользователей}
  \item $c_Y(i) \bigcap c_Y(i^{\prime}) = c_Y(i^{\prime \prime}): \forall y \in Y$ $\mu_u(y) =
	  \min(\mu_{i^{\prime}}(y), \mu_{ i^{\prime \prime}}(y))$ --- {\it
		  пересечение контентов объектов}
	  \item $c_X(u) \bigcap c_X(u^{\prime}) = c_X(u^{\prime \prime}): \forall x \in X$ $\mu_u(x) =
	  \max(\mu_{u^{\prime}}(x), \mu_{ u^{\prime \prime}}(x))$ --- {\it пересечение контентов пользователей}
  \item $c_Y(i) \bigcap c_Y(i^{\prime}) = c_Y(i^{\prime \prime}): \forall y \in Y$ $\mu_u(y) =
	  \max(\mu_{i^{\prime}}(y), \mu_{ i^{\prime \prime}}(y))$ --- {\it
		  пересечение контентов объектов}
  \end{itemize}
\end{frame}


\begin{frame}
  \frametitle{Нечеткая модель. Отношение сходства, расстояние}
    \scriptsize{
  \begin{variableblock}{Обобщенное расстояние Хэмминга между пользователями}{
	  }{bg=green!9,fg=blue}
    \begin{equation}		\overset{u}{\mathcal{\rho}}(u,u^{\prime}) =
      \begin{cases}
		  \text{Не определено, если} & c_X(u) \bigcap c_X(u^{\prime}) = \varnothing\\
		  \frac{1}{|X|} \cdot \sum \limits_{k=1}^{|X|} |\mu_u(x_k) -
		  \mu_{u^{\prime}}(x_k)| & \text{иначе}
      \end{cases}
    \end{equation}
  \end{variableblock}

  \begin{variableblock}{Обобщенное расстояние Хэмминга между объектами}{
	  }{bg=green!9,fg=blue}
    \begin{equation}
		\overset{i}{\rho}(i,i^{\prime}) =
      \begin{cases}
         \text{Не определено, если} & c_Y(i) \bigcap
		  c_Y(i^{\prime}) = \varnothing \\
        \frac{1}{|Y|} \cdot \sum \limits_{k=1}^{|Y|} |\nu_i(y_k) - \nu_j(y_k)| & \text{иначе}
      \end{cases}
    \end{equation}
  \end{variableblock}

	 \begin{variableblock}{Отношение сходства пользователей. Обладает свойством
		 транзитивности}{
		 }{bg=green!9,fg=blue}
     \begin{equation}\label{ru}
		 u \ru u^{\prime} \Leftrightarrow \overset{u}{\mathcal{\rho}}(u,u^{\prime})\le \epsilon_0 \in \varepsilon(0)
     \end{equation}
   \end{variableblock}

	 \begin{variableblock}{Отношение сходства объектов. Обладает свойством
		 транзитивности.}{
		 }{bg=green!9,fg=blue}
     \begin{equation}\label{rt}
	   i \rt i^{\prime} \Leftrightarrow \overset{i}{\rho}(i,i^{\prime}) \le \epsilon_0 \in \varepsilon(0)
     \end{equation}
  \end{variableblock}
}
\end{frame}

\begin{frame}
  \frametitle{Нечеткая модель. Отношение сходства пользователя и объекта,
	нечеткое правило выводы $\Pi_f$}
	\scriptsize{
  \begin{variableblock}{Отображение пользователей на множество объектов}{ }{bg=green!9,fg=blue}
	\begin{itemize}
	\item Оценка сходства характеристик: $\delta_c : X \times Y \Rightarrow [0,1]$
	\item Нечеткое отображение $\Psi : X \rightarrow Y$
	\item
		\begin{equation}
				\nu_{\Psi}(y) = \underset{x \in X} {\mathrm{\sup}} \min\{ \rho_c(x,y); \mu(x) \}
		\end{equation}

	\item
		\begin{equation}
			\Psi(c_X(u)) = \{ (y | \nu_{\Psi}(y)) \}, y \in Y
		\end{equation}
	\end{itemize}
  \end{variableblock}

  \begin{variableblock}{Расстояние между пользователем и объектом}{ }{bg=green!9,fg=blue}
		\begin{equation}
			\overline{\rho}(u,i) = \overset{i}{\mathcal{\rho}}(\overline{i}, i)\text{, где }
			c_Y(\overline{i}) = \Psi(c_X(u))
		\end{equation}
  \end{variableblock}

  \begin{variableblock}{Отношение сходства пользователя и объекта. Обладает
	  свойством транзитивности}{ }{bg=green!9,fg=blue}
		\begin{equation}
			u \mathit{R} i\Leftrightarrow \text{ } \overline{\rho}(u,i) \le \epsilon
		\end{equation}
  \end{variableblock}

	\begin{variableblock}{Для решения задач производится расчет значения
		$\rho(u_a,i)$}{ }{bg=green!9,fg=blue}
		  \begin{equation}
			  \Pi_f = \overline{\rho}(u_a,i)
		  \end{equation}
	\end{variableblock}
}
\end{frame}

\begin{frame}
	\frametitle{Нечеткая модель}
	\begin{equation}
		(c_X, c_Y, \Pi \in \{\Pi_{OOM}, \Pi_{COM}, \Pi_f\}),
	\end{equation}
	где $c_X, c_Y$ --- нечеткие подмножества.
\end{frame}

\begin{frame}
	\frametitle{Нечеткая модель как эффективное расширение}
	\begin{variableblock}{По критерию качества решения}{ }{bg=green!9,fg=blue}
	\begin{columns}[T]
		\column{.5\textwidth} % Left column and width
		\begin{variableblock}{$OOM$}{ }{bg=green!9,fg=blue}
			Решение:

			Обладает большей эффективностью, так как выполняется достаточное
			условие эффективности решения.
		\end{variableblock}

		\column{.5\textwidth} % Left column and width
		\begin{variableblock}{$COM$}{ }{bg=green!9,fg=blue}
			Решение:
			Обладает большей эффективностью, так как выполняется достаточное
			условие эффективности решения.
		\end{variableblock}
	\end{columns}
	\end{variableblock}

	\begin{variableblock}{По критерию вычислительной сложности}{ }{bg=green!9,fg=blue}
	\end{variableblock}
\end{frame}

\begin{frame}
\frametitle{Расширение модели для задачи прогнозирования по параметру
	порогового
	значения}
\scriptsize {
	\begin{variableblock}{Решение подзадачв ОРС и СРС. Множество соседей}{ }{bg=green!9,fg=blue}
		\begin{enumerate}
			\item  СРС: $\mathit{N} = \{ u : a \ru u \wedge \exists \text{ }
		\rho(u,i_{\times}) \in P\}$ \\
	\item  ОРС: $\mathit{N} = \{ i : a \rt \{i_0\}\}$, $\rho(u_a, i_0) \in T^a_0$
	\end{enumerate}
	\end{variableblock}

	\begin{variableblock}{Условие вхождение пользователя в кластер}{ }{bg=green!9,fg=blue}
		\begin{equation*}\label{method}
			e \mathit{R} c \wedge e \mathit{R} c^{\bigcup} \wedge e \mathit{R} c^{\bigcap},
		\end{equation*}
		\begin{itemize}
			\item $e$ --- новый элемент кластера, либо пользователь $u$ (для
				СРС), либо объект $i$ (для ОРС);
			\item $c$ --- центр множества соседей.
			\item $c^{\bigcup} = e^{\prime} \bigcup e^{\prime \prime} \bigcup ...
				\bigcup u$, $e^{\prime},e^{\prime \prime},... \in \mathit{N}$;
			\item $c^{\bigcap} = e^{\prime} \bigcap e^{\prime \prime} \bigcap ...
				\bigcap e$.
		\end{itemize}

		\begin{assert}
			Введенная модификация алгоритма составления кластера гарантирует
			выполнение транзитивности отношения $\ru$ или $\rt$ на множестве соседей
			при $\epsilon_0 \ge \epsilon \in \varepsilon(0)$
		\end{assert}
	\end{variableblock}
	}
\end{frame}

\begin{frame}
  \frametitle{Эффективный алгоритм составления множества соседей на примере СРС}
%\begin{center}
%  \includegraphics[width=3in,height=3in]{pics/alg-neighb.png}
%\end{center}
\end{frame}


\begin{frame}
  \frametitle{Результат}
\begin{variableblock}{Нечеткая коллаборативная РС}{ }{bg=green!9,fg=blue}
{\it Нечеткая коллаборативная } РС --- РС типа $(X, Y, P, \Pi,
	\mathcal{E})$, где $\Pi$:
		\begin{itemize}
			\item $(u_a \R i)$ $\wedge$
				$(\overset{i}{\rho}(i,i^{\prime}) \le \epsilon_0) \Rightarrow$
				$a \R i^{\prime}$;
			\item $\overset{u}{\rho}(u_a, u) \le \epsilon_0 \text{ для } T_0 \Rightarrow 
							\overset{u}{\rho}(u_a, u) \le \epsilon_0  \text{ для } T^a_{\times}$;
			\item $c_X(u), c_Y(i)$ --- нечеткие множества.
		\end{itemize}
\end{variableblock}

\begin{itemize}
\item Определены отношения сходства сходства элементов системы, обладающие
	свойством транзитивности, решена задача (2);
\item В общем случае, нечеткая коллаборативная РС эффективней
	коллаборативной, так как выполняется условие эффективности решений:
		транзитивность отношений сходства.
\end{itemize}
\end{frame}







\begin{frame}
  \frametitle{Оценка эффективности. Предпосылка}
  \begin{variableblock}{}{ }{bg=green!9,fg=blue}
	\begin{center}
		{\bf Контент реального пользователя}:
		\begin{itemize}
		\item $c_X(u) = \{(1 | \rho(u, 1)), (2 | \rho(u, 2)), ..., (m | \rho(u, m))\}$
		\item $\rho(k, i)$ --- выступает в роли характеристической функции
		\item $\mathbb{N}^m$ --- множество характеристик
		\end{itemize}

	  {\bf Прототип контента реального пользователя, сформированный РС:}
		\begin{itemize}
			\item $c_X(\overline{u}) = \{(1 | \overline{\rho}(u, 1)), (2 | \overline{\rho}(u, 2)), ..., (m | \overline{\rho}(u, m))\}$
		\item $\overline{\rho}(k, i)$ --- выступает в роли характеристической
			функции. Для $\rho(u, i) \in P$, $\overline{\rho}(u, i)) =
				\rho(u, i)$;
		\item $\mathbb{N}^m$ --- множество характеристик
		\end{itemize}

	  {\bf Решение эффективно, если:}
		\begin{itemize}
		\item $\overset{u}{\mathcal{\rho}}(u, \overline{u}) \le \epsilon$
		\end{itemize}
	\end{center}
\end{variableblock}
\end{frame}


\begin{frame}
  \frametitle{Оценка эффективности РС}
	\begin{variableblock}{Определение:}{ }{bg=green!9,fg=blue}
		\begin{center}
		$\mathcal{E}(T^a_{\times}, \overline{T}_{\times}) =
			\frac{1}{|N|} \cdot \sum \limits_{
				i \in \overline{T^a_{\times}},
				i^{\prime} \in T^a_{\times}
			} |\overline{\rho}(u_a, i) - \rho(u_a,
			i^{\prime})| $, $T^a_{\times}$,
			$\overline{T}_{\times}$ упорядочены по возрастанию.
		\end{center}
	\end{variableblock}

	\begin{variableblock}{Свойства оценки эффективности}{ }{bg=green!9,fg=blue}
		\begin{itemize}
				\scriptsize{
			\item Обладает метрическими свойствами, а, значит ее значение
				может использоваться в качестве количественного показателя
				эффективности
			\item Может применяться к подзадаче прогнозирования и $\top$
			\item Коррелирует с оценками эффектиности, принадлежащих классу
				оценок подзадачи проггнозирования и $\top$
			\item Применение не зависит от свойств исходных данных.
				}
		\end{itemize}

	\end{variableblock}
	\begin{center}
		\includegraphics[width=3in,height=1in]{pics/alg-e.pdf}
	\end{center}
\end{frame}

\begin{frame}
	\frametitle{Решения задач РС}
	\scriptsize{
		\begin{columns}[T]
			\column{.5\textwidth} % Left column and width
			\begin{variableblock}{}{ }{bg=green!9,fg=blue}
				Задача $\top$
			\end{variableblock}
			\begin{center}
				\includegraphics[width=2in,height=1.5in]{pics/alg-topn.pdf}
			\end{center}

			\column{.5\textwidth} % Left column and width
			\begin{variableblock}{}{ }{bg=green!9,fg=blue}
				Задача прогнозирования
			\end{variableblock}

			\begin{center}
				\includegraphics[width=1.5in,height=1in]{pics/alg-p.pdf}
			\end{center}
		\end{columns}

	\begin{variableblock}{Свойства решений}{ }{bg=green!9,fg=blue}
		\begin{itemize}
		\item Решения оптимальны по критерию $\mathcal{E}$.
		\item Решения не зависят от $T_0$ (или $P$), так как расчет
			расстояния производится на основании $X$, $Y$.
		\end{itemize}
	\end{variableblock}
	}
\end{frame}



\begin{frame}
  \frametitle{Сравнение решений}
	\scriptsize{
\begin{table}[h]
		  \begin{center}
			\begin{tabular}{|c|c|c|}
			  \hline
			  Свойство решений & Двумерная модель & Контентная модель   \\ \hline
				Требуется $P$  & Да & Нет   \\ \hline
				Зависимость от динамики  & & \\ и неоднородности $P$ и $P_{\times}$  & Да & Нет   \\ \hline
				Зависимость от $X$ и $Y$  & Нет & Да   \\ \hline
				Реализация $\rho: \mathbb{N}^m \times N^{n} \rightarrow [0,1]$
				& Косвенная, основана на эвристике & Формальная \\ \hline
			\end{tabular}
		  \end{center}
		\end{table}
 \begin{center}
	\begin{table}[h]
	  \begin{tabular}{|c|c|c|c|}
		\hline
		Модель & Задача & Сложность  \\ \hline
		Двумерная & $\top$ & $O(|T|^2)$  \\ \hline
		Двумерная & Прогнозирование & $O(|U|)$ \\ \hline
		Контентная & $\top$ & $O(|T|)$ \\ \hline
		Контентная & Прогнозирование & $O(C)$ \\ \hline
	  \end{tabular}
	\end{table}
	\end{center}
}
\end{frame}


\begin{frame}
  \frametitle{Результат}
\begin{itemize}
	\item Определены формальные $\Pi$, на базе которых разработаны эффективные
		решения, не зависящие от $T_0$\footnote{С. А. Амелькин, Д. П. Понизовкин, Математическая модель задачи
top-N для контентных рекомендательных систем,
Известия МГТУ МАМИ, 2 , c. 26–31}. Решена задача (4);
\item Разработана контентная модель, являющаяся эффективным расширением
	двумерной модели;
\item Введена оценка эффективности, которая:
	\begin{itemize}
	\item Является метрикой, а, значит, может использоваться в качестве
		количественного показателя эффективности;
	\item Является расстоянием между реальным решением и сформированным РС, а,
		значит, показатель объективен и соответствует подазадачам;
	\item Коррелирует с существующими оценками;
	\item Не зависит от свойств $T_0$, может быт применена к любой подазадаче.
	\end{itemize}
		Решена задача (3);
\end{itemize}
\end{frame}



\begin{frame}
	\frametitle{Апробация работы}
	\begin{columns}[T]
		\column{.6\textwidth} % Left column and width
		\begin{itemize}
				\scriptsize{
				\item $|\mathbb{N}^m|$ = 6000;
				\item $|\mathbb{N}^n|$ = 10000 --- множество объектов, являюхся
					ффильмами ильмов;
				\item $|\{ \rho(u, i) \in P \} |$ = 1 000 000;
				\item $|Y| = 18$ --- множество кинематографических жанров;
				\item $X$ --- оценки пользователей для коллаборативной модели;
				\item $X = \{i\} \subset \mathbb{N}^n$, $\rho(u,i) \in P$;
				\item $\delta_c(i, y) = |\{ i : (\rho(u, i) < \rho_1) \wedge (\nu(y) = 1)\}| -
					|\{ i : (\rho(u, i) > \rho_2) \wedge (\nu(y) = 1)\}|$,
				$\rho = 0.2$, $\rho_2 = 0.6$;
				Предполагается, что между оценкой пользователя и жанрами, характерными для объекта существует корреляция.
				\item $\nu_l(y_i) \in \{0,1\}$
					}
		\end{itemize}
		\column{.4\textwidth} % Left column and width
		\begin{center}
			\includegraphics[width=1.5in,height=2in]{pics/Content_recomendat_poisk.pdf}
		\end{center}
	\end{columns}
\end{frame}



\begin{frame}
  \frametitle{Эффективность расширения по параметру функции, использующейся для
	копределения отношения сходства и порогового значения}
\scriptsize{
	\begin{columns}[T]
		\column{.6\textwidth} % Left column and width
		\begin{variableblock}{Подзадача $\top$. Расширение по функции определния
			отношения сходства}{ }{bg=green!9,fg=blue}
			\begin{table}[h]
				\begin{center}
					\begin{tabular}{|c|c|c|c|c|}
						\hline
						Модель & P & AveP & NDCG  \\ \hline
						Двумерная & 0.0105 & 0.011  & 0.0419  \\ \hline
						Контентная & 0.0004 &  0.0004 & 0.0318  \\ \hline
					\end{tabular}
				\end{center}
			\end{table}
		\end{variableblock}

		\column{.5\textwidth} % Left column and width
	\begin{variableblock}{Подзадача прогнозирования. Расширение по функции определния
			отношения сходства}{ }{bg=green!9,fg=blue}
		\begin{table}[h]
			\begin{center}
				\begin{tabular}{|c|c|c|}
					\hline
					Модель      & MAE & RMSE \\ \hline
					Двумерная & 0.168 & 0.222  \\ \hline
					Контентная & 0.155  & 0.213  \\ \hline
				\end{tabular}
			\end{center}
		\end{table}
	\end{variableblock}
	\end{columns}

	\begin{variableblock}{Подзадача прогнозирования. Расширение по функции определния
			отношения сходства и пороговому значению}{ }{bg=green!9,fg=blue}
	  \begin{table}[h]
	\begin{center}
	  \begin{tabular}{|c|c|c|}
		\hline
		Тип & MAE & RMSE \\ \hline
		Двумерная & 0.168 & 0.222  \\ \hline
		 Контентная & 0.162 & 0.214 \\ \hline
	  \end{tabular}
	\end{center}
  \end{table}
  \end{variableblock}
	%----------------------------------------------------------------
}
\end{frame}


\begin{frame}
  \frametitle{Эффективность контентной модели}
  \begin{variableblock}{Подзадача $\top$}{ }{bg=green!9,fg=blue}
    \begin{table}[h]
      \begin{center}
		\scriptsize{
        \begin{tabular}{|c|c|c|c|c|}
          \hline
          Разбиение/&&&&\\Модель & P & AveP & NDCG & $\mathcal{E}$ \\ \hline
          80 к 20/Двумерная & 0.0105 & 0.011  & 0.0419 & 0.57 \\ \hline
          80 к 20/Контентная &&&&\\модель & 0.0472 & 0.053 & 0.0693 & 0.079 \\ \hline
          40 к 60/Двумерная & 0.5619 & 0.5986  & 0.5921 & 0.7114 \\ \hline
          40 к 60/Контентная&&&&\\модель & 0.0472 & 0.053 & 0.0693 & 0.079 \\ \hline
          Случайное &&&&\\ решение  & 0.7141& 0.7994 & 0.7541 & 0.5632 \\ \hline
        \end{tabular}
		  }
      \end{center}
    \end{table}
  \end{variableblock}

\scriptsize{
\begin{variableblock}{Подзадача прогнозирования}{ }{bg=green!9,fg=blue}
  \begin{table}[h]
	\begin{center}
	  \begin{tabular}{|c|c|c|}
		\hline
		Разбиение/Модель & MAE & RMSE \\ \hline
		80 к 20/Двумерная& 0.164 & 0.215  \\ \hline
		80 к 20/Контентная& 0.183 & 0.247  \\ \hline
		40 к 60/Двумерная &  0.191  & 0.25  \\ \hline
		40 к 60/Контентная& 0.183 & 0.247  \\ \hline
		Случайное решение  & 0.253 & 0.31 \\ \hline
	  \end{tabular}
	\end{center}
  \end{table}
\end{variableblock}
}
\end{frame}


\begin{frame}
  \frametitle{Выводы}
  \begin{itemize}
  \scriptsize{
  \item Проведено исследованик двумерных моделей в ходе которого определены:
    \begin{itemize}
	\scriptsize{
    \item Необходимое и достаточные условия применения двумерных
		моделей;
    \item Необходимое и достаточные условия эффективности
		решений подзадач в двумерных моделях;
		}
    \end{itemize}
  \item Разработана математическая модель контентной РС, которая:
        \begin{itemize}
		\scriptsize{
        \item является эффективным расширением двумерной модели;
        \item разработана обощенная оценка эффективности;
        \item разработаны эффективные решения подзадач, алгоритмы которых
			обладают меньшей асимптотической сложностью по сравнению с
				двумерными;
				}
        \end{itemize}
  \item Разработано зарегистрированное ПО, реализующее двумерную и контентную модели.
	  }
  \end{itemize}
\end{frame}

\begin{frame}
  \frametitle{Опубликованные статьи}
  \scriptsize{
\begin{enumerate}
\item Понизовкин Д. М. Оптимальное распределение проектов при проведении экспертизы / Д. М. Понизовкин, С. А. Амелькин // 
Электронные библиотеки: Перспективные Методы и Технологии, Электронные коллекции. --- 2010. --- С. 524-525.
\item Понизовкин Д. М. Построение оптимального графа связей в системах коллаборативной фильтрации / Д. М. Понизовкин, С. А. Амелькин // 
Программные системы: теория и приложения. 2011.--- Т. 2. --- № 4. С. 107–114
\item Понизовкин Д. М. Математическая модель коллаборативных процессов принятия решений / Д. М. Понизовкин, С. А. Амелькин // 
Программные системы: теория и приложения. 2011. --- Т. 2. --- № 4. С 95-99.
\item Амелькин С. А. Оптимальное проведение экспертизы образовательных процессов / С. А. Амелькин, Д. М. Понизовкин // 
Труды XVII Всероссийской научно-методической конференции Телематика’2010, Санкт-Петербург: Университетские телекоммуникации. --- 2010. ---
№ 1, С. 158-159.
\item Д. М. Понизовкин. Влияние меры сходства на результативность РС // Программные системы: теория и приложения, 2014. --- т. 2. --- N. 5. С 55–65.
\item С. А. Амелькин. Математическая модель задачи top-N для контентных рекомендательных систем / С. А. Амелькин., Д. П. Понизовкин //
Известия МГТУ МАМИ, 2 , c. 26–31
\end{enumerate}
}
\end{frame}

\begin{frame}
  \frametitle{Статьи в печати}
  \scriptsize{
    \begin{enumerate}
    \item Понизовкин Д. М. Повышение эффективности решения задачи прогнозирования в эвристических коллаборативных рекомендательных системах / Д. М. Понизовкин //
      Искусственный Интеллект и Принятие Решений.
    \item Понизовкин Д. М. Повышение эффективности решения задачи top-N в эвристических коллаборативных рекомендательных системах / Д. М. Понизовкин // 
      Информационные технологии
    \item Понизовкин Д. М. Применение точности как оценки эффективности решения задачи top-N в объектных коллаборативных рекомендательных система / 
      Д. М. Понизовкин, С. А. Амелькин // Программная инженерия
    \item Понизовкин Д. М. Модели рекомендательных систем, основанные на теории нечетких множеств / Д. М. Понизовкин // Искусственный Интеллект и Принятие Решений.
    \end{enumerate}
  }
\end{frame}

\frame{\titlepage}
%%%%%%% END
\end{document}
