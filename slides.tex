%\documentclass[10pt,pdf,hyperref={unicode},fleqn]{beamer}
\documentclass[10pt,xcolor={usenames,dvipsnames}]{beamer}

\usepackage[T2A]{fontenc}       %поддержка кириллицы
\usepackage[utf8]{inputenc}
\usepackage{longtable} 
\usepackage{cancel}
\usepackage{amsmath}
\usepackage{tikz}
\usepackage{graphicx}

\usetheme{Darmstadt}
\setbeamercolor{block title}{bg=white!30,fg=black}
\newenvironment{variableblock}[3]{%
  \setbeamercolor{block body}{#2}
  \setbeamercolor{block title}{#3}
  \begin{block}{#1}}{\end{block}}

%% \usefonttheme{structurebold}
\usefonttheme[onlymath]{serif}
\setbeamertemplate{footline}{\hspace*{.5cm}\scriptsize{
\hspace*{50pt} \hfill\insertframenumber/\inserttotalframenumber\hspace*{.5cm}}} 
%\setbeamercovered{transparent}
%\setbeamercolor{color1}{bg=blue!90!black,fg=white}
\setbeamercolor{normal text}{bg=white,fg=black}
\setbeamercolor{frametitle}{fg=black,bg=gray}
%\insertframenumber
%% \setbeamercolor{footline}{fg=blue}
%% \setbeamerfont{footline}{series=\bfseries}
%\setbeamerfont{page number in head/foot}{size=\small}
%\hfill\insertframenumber/\inserttotalframenumber
%\setbeamerfont{page number in head/foot}{size=\large}
\theoremstyle{break}
\newtheorem{assert}{Утверждение}[section]
\newtheorem{assertORS}{Утверждение ОРС}
\newtheorem{assertSRS}{Утверждение СРС}
\newtheorem{myproof}{Доказательство}[section]

%% \defbeamertemplate*{footline}{shadow theme}
%% {%
%%   \leavevmode%
%%   \hbox{\begin{beamercolorbox}[wd=.5\paperwidth,ht=2.5ex,dp=1.125ex,leftskip=.3cm plus1fil,rightskip=.3cm]{author in head/foot}%
%%     \usebeamerfont{author in head/foot}\insertframenumber\,/\,\inserttotalframenumber\hfill\insertshortauthor
%%   \end{beamercolorbox}%
%%   \begin{beamercolorbox}[wd=.5\paperwidth,ht=2.5ex,dp=1.125ex,leftskip=.3cm,rightskip=.3cm plus1fil]{title in head/foot}%
%%     \usebeamerfont{title in head/foot}\insertshorttitle%
%%   \end{beamercolorbox}}%
%%   \vskip0pt%
%% }
\def\top{\text{top-N}}
\def\p{\text{p}}

\begin{document}

%1
\title{ИПС им. А. К. Айламазяна РАН \\ Исследовательский Центр системного Анализа}
\author{Понизовкин Д. М.  \\ Научный руководитель: к. т. н. Амелькин С. А. } 
\institute{{\Large РАЗРАБОТКА МАТЕМАТИЧЕСКОЙ МОДЕЛИ И АЛГОРИТМОВ ИНФОРМАЦИОННЫХ ПРОЦЕССАХ В РЕКОМЕНДАТЕЛЬНЫХ СИСТЕМАХ.}}

% \institute{XVIII ежегодная молодежная научно-практическая конференция «Наукоемкие информационные технологии»}
% \date[conf-name]{conference long name 2013}
% \date{27.11.2015} 

\frame{\titlepage} 

%2: Примеры.
\begin{frame}
    \frametitle{Рекомендательные системы (далее РС) в общем.}
    \begin{columns}[T]
    \column{.5\textwidth} % Left column and width
    \begin{block}{}
      \includegraphics[width=2in,height=2in]{pics/rs.pdf}
    \end{block}

    \column{.5\textwidth} % Left column and width
    \begin{block}{Примеры известных РС}
    \begin{center}
      \begin{itemize}
      \item Новостные: 
        \begin{itemize}
        \item Google News
        \item Yahoo! News 
        \end{itemize}
      \item Музыкальные:

        \begin{itemize}
        \item LastFm
        \item Spotify
        \end{itemize}
        
      \item Товарные:
        \begin{itemize}
        \item Amazon
        \item Ebay
        \end{itemize}

      \item Кинематографические:
        \begin{itemize}
        \item Netflix
        \item MovieLens
        \end{itemize}
      \end{itemize}
    \end{center}
    \end{block}
  \end{columns}
\end{frame}

%% \begin{frame}
%%   \frametitle{Кинематографический сервис. Прогноз оценки. Movie Lens.}
%%   \begin{center}
%%     \includegraphics[width=3in,height=1in]{pics/movielens.jpg}
%%   \end{center}
%% \end{frame}

\begin{frame}
  \frametitle{Задачи РС.}
    \begin{columns}[T]
    \column{.5\textwidth} % Left column and width
    \begin{variableblock}{$\top$}{bg=SpringGreen,fg=black}{bg=blue,fg=white}
      Определить $N$ объектов с высокой степенью предпочтения пользователя.
    \end{variableblock}

    \column{.5\textwidth} % Left column and width
    \begin{variableblock}{Прогнозирование.}{bg=SpringGreen,fg=black}{bg=blue,fg=white}
      Определить степень предпочтения пользователя конкретного объекта.
    \end{variableblock}
  \end{columns}
\end{frame}

\begin{frame}
  \frametitle{Классификация РС.}
  \begin{itemize}
  \item Коллаборативные: 
    \begin{enumerate}
    \item Объектно-ориентированные, далее ОРС;
    \item Субъектно-ориентированные, далее СРС;
    \end{enumerate}
  \item Кластерные;
  \item Статистические: латентный анализ, факторный анализ, цепи Маркова, наивный Байесовский классификатор и т.д;
  \item Машинное обучение;
  \item Контентные;
  \end{itemize}
\end{frame}

\begin{frame}
  \frametitle{Основные задачи исследования}
\begin{itemize}
\item Ввести математическую модель РС:
  \begin{itemize}
    \item Задать модель представления данных;
    \item Задать основные отношения между элементами системы;
  \end{itemize}
\item Ввести объективный способ оценки эффективности, сооветствующий задачам РС;
%\item Ввести модель РС, в которой основа --- формальная дисциплина, а не эвристика.
%\item Ввести модель РС, в которой возможно оперировать с информацией, обладающей свойством неопределенности, характерной для гуманистических информационных систем.
%\item Ввести модель РС, в которой формализовать КРС и повысить их эффективность.
\item Ввести решения задач, которые:
  \begin{enumerate}
  \item дают эффективное решение по введенной оценке эффективности.
  \item описываются алгоритмами, обладающими свойством масштабируемости.
  \item не зависят от дополнительных ограничений, чем гарантируется устойчивость эффективного решения.
  \item предоставляют как минимум сравнимый результат, получаемый при решении задач в других моделях.
  \end{enumerate}
%\item Определить функцию оценки сходства между пользователем и объектом такую;
%\item Ввести оценку эффективности, удовлетворяющую целевым задачам РС и коррелирующую с существующими оценками эффективности.
\item Разработать ПО для проведения тестирования.
\item Провести тесты на реальных данных, сравнить результаты, полученные при решении задач во введенной модели и
в одной из существующей и используемой медели.
\end{itemize}
\end{frame}


%2 http://www.slideshare.net/huguk/collaborative-filtering-at-scale-2-7671787
% http://www.slideshare.net/abellogin/improving-memorybased-collaborative-filtering-by-neighbour-selection-based-on-user-preference-overlap

% 10
\begin{frame}
\frametitle{Определения и обозначения.}
\small {
\begin{block}{}
  \begin{itemize}
    \item $U$ --- множество {\it пользователей}, $I$ --- множество {\it объектов}.
    \item $[0,1]$ --- относительная шкала оценки сходства пользователя и объекта.
    \item $\delta_0(u, i)$ --- функция оценки сходства, поставленное пользователем $u$ объекту $i$.
    \item $\delta_0(u, i) = \varnothing$ --- значение неизвестно.
    \item $\delta_1(u, i)$ --- функция оценки сходства, вычисленное системой.
  \end{itemize}

  \begin{center}
    $\delta: U \times I \rightarrow [0,1], \delta \in \{\delta_0, \delta_1\}$ 
  \end{center}

  \begin{center}
    $\delta_1: U \times U \rightarrow [0,1]$ 
  \end{center}

  \begin{center}
    $\delta_1: I \times I \rightarrow [0,1]$ 
  \end{center}
\end{block}
}
\end{frame}

%11
%\underset{ \{ v^k \} } {\mathrm{max}}
\begin{frame}
%  \scriptsize {
    \frametitle{Терминология и обозначения}
    \begin{enumerate}
    \item {\bf Контент} пользователя $u \in U$ или объекта $i \in I$ --- информация о характеристиках пользователя $u$ или объекта $i$.
    \item {\bf Отношение сходства} --- $e^1 \mathit{R} e^2 \Leftrightarrow \delta_0(e^1, e^2) > \Delta$ или $\delta_1(e^1, e^2) > \Delta$. $e^1, e^2$ --- {\bf соседи}.
    \item {\bf Оценка эффективности} $\mathcal{E} : T^a_{1} \times T^a_{\times} \rightarrow [0,1]$.
      \begin{itemize}
      \item $T^a, T^a_0, T^a_{\times}, T^a_1$ --- входное, обучающее, тестовое и результирующее множества.
      \item $T^a_{0} \bigcup T^a_{\times} = T^a$, $T^a_{0} \bigcap T^a_{\times} = \varnothing$.
      \item $T^a = I \times \{ [0,1] \bigcup \varnothing \}$: \\
        $\{ (i, \delta_0(a, i)) \}$, $\delta_0(a, i) = \varnothing \Rightarrow$ $\bcancel{ \exists }$  $\delta_0(a, i)$.
      \item $a$ --- активный пользователь, для которого решается задача.
      \end{itemize}
    \end{enumerate}

\begin{variableblock}{}{bg=SpringGreen,fg=black}{bg=blue,fg=white}
\begin{center}
Чем меньше значение $\mathcal{E}$, тем решение эффективней.
\end{center}
\end{variableblock}

%  } 
\end{frame}

%12
\begin{frame}
  \frametitle{Целевые задачи РС}
    \begin{columns}[T]
    \column{.5\textwidth} % Left column and width
    \begin{variableblock}{$\top$}{bg=SpringGreen,fg=black}{bg=blue,fg=white}
      $T^a_{1} = \{ (i_{1},\delta_1(a,i_{1}))\}$, где $a \mathit{R} i_{1}$ $\wedge$ $|T^a_{1}| = N$
    \end{variableblock}

    \column{.5\textwidth} % Left column and width
    \begin{variableblock}{Прогнозирование.}{bg=SpringGreen,fg=black}{bg=blue,fg=white}
      $T^a_1 = \{ (i_{\times},\delta_{1}(a,i_{\times}))\}$, где $|\delta_{1}(a,i_{\times}) - \delta_0(a,i_{\times})| \le \epsilon_0 \in \varepsilon(0)$
    \end{variableblock}
  \end{columns}

    \begin{center}
      $\underset{T^a_{1}} {\mathrm{argmin}}$ $\mathcal{E}(T^a_{1}, T^a_{\times})$
    \end{center}
\end{frame}

\begin{frame}
  \frametitle{Целевые оценки эффективности}
    \begin{columns}[T]
    \column{.5\textwidth} % Left column and width
    \begin{variableblock}{$\top$}{bg=SpringGreen,fg=black}{bg=blue,fg=white}
      $\mathcal{E}^a_{t} = 1 - \frac{1}{N} \cdot \sum \limits_{i_{1} \in T^a_1} s(n)$,\\$
s(n) = 
\begin{cases}
1, &\text{$a  \mathit{R} i^n_{1}, n=1..N$}\\ %, $i^n_1 \in I_{1}^a, i^n \in I_{0}^a$}\\
0, &\text{иначе}.
\end{cases}
$
    \end{variableblock}

    \column{.5\textwidth} % Left column and width
    \begin{variableblock}{Прогнозирование}{bg=SpringGreen,fg=black}{bg=blue,fg=white}
      $\mathcal{E}^a_{p} = \sum \limits_{i_{1} \in T^a_1} |\delta_0(a,i_{1}) - \delta_1(a,i_{1})|$
    \end{variableblock}
  \end{columns}
\end{frame}


%% \begin{frame}
%% \frametitle{Рекомендательные системы. Формально. Задачи}
%% \begin{block}{\textcolor{blue}{$\top$}}
%%   $I' \subset I:$ $\{ i \in I' : \delta_0(a, i) > \Delta, $ $\delta_0(a, i) = \varnothing$ $\}$, $|I'| = N$, $a$ --- активный пользователь
%% \end{block}

%% \begin{block}{\textcolor{blue}{Прогнозирование}}
%%   Вычисление $\delta_1(u^a, t_{\times}) : $ $|\delta_1(u^a, t_{\times}) - \delta_0(u^a, t_{\times})| \le \epsilon_0$
%% \end{block}
%% \end{frame}

%%%%%%%%%%%%%%%%%%%%%%%%%%%%%
% chapter: fuzzy
%%%%%%%%%%%%%%%%%%%%%%%%%%%%%%
\begin{frame}
  \frametitle{Представление контентов на базе теории нечетких множеств}
  \begin{itemize}
  \item $u^m = \{(x_i | \mu_m(x_i )) \}, \mu_m(x_i) \in [0,1], x \subseteq X$
  \item $i^n = \{(y_i | \nu_n(y_i )) \}, \nu_n(y_i) \in [0,1], y \subseteq Y$
  \item $u^1 \bigcap u^2 = u^3:$ $\forall$ $x \in X$ $\mu_3(c) = \min(\mu_1(x), \mu_2(x))$
  \item $i^1 \bigcap i^2 = i^3:$ $\forall$ $y \in Y$ $\nu_3(c) = \min(\nu_1(x), \nu_2(x))$
  \item {\it Пустой контент} $\varnothing$:  $\forall$ $c \in C$, $f(x) = 0, C \in \{X, Y\}, f \in \{\nu, \mu\}$
  \item {\it Пересечение контентов} $e^1 \bigcap e^2 = e^3: \forall c \in C$ $f_3(x) = \min(f_1(x), f_2(x)), C \in \{X, Y\}, f \in \{\nu, \mu\}$
  \item {\it Объединение контентов} $e^1 \bigcap e^2 = e^3: \forall c \in C$ $f_3(x) = \max(f_1(x), f_2(x)), C \in \{X, Y\}, f \in \{\nu, \mu\}$
%  \item Контенты {\it сравнимы}, если их пересечение не является пустым контентом
  \end{itemize}
\end{frame}

\begin{frame}
  \frametitle{Отношение сходства во введенной модели}

  \begin{block}{Обобщенное расстояние Хэмминга}
    \begin{equation*}
      \rho(e^1,e^2) = 
      \begin{cases}
        \varnothing, & e^1 \bigcap e^2 = \varnothing\\
        \frac{1}{|C|} \cdot \sum \limits_{l=1}^{|C|} |f_1(c_l) - f_2(c_l)| \ne 0 & \text{иначе}, f_i(c_l)
      \end{cases}
    \end{equation*}
  \end{block}

  \begin{block}{Отношение сходства}
    \begin{equation}\label{rhoSim}
      e_1\mathit{R}e_2 \Leftrightarrow \rho(e_1, e_2) \le \epsilon_0 \in \varepsilon(0)
    \end{equation}
  \end{block}

  \begin{block}{Взаимосвязь расстояния и оценки сходства}
    \begin{equation}
      \delta_0(e_1, e_2) = 1 - \rho(e_1, e_2)
    \end{equation}
  \end{block}

\end{frame}

\begin{frame}
  \frametitle{Отношение сходства во введенной модели. Свойства}
\begin{variableblock}{$\rho(e_1, e_2)$ обладает метрическими свойствами}{bg=SpringGreen,fg=black}{bg=SpringGreen,fg=red}  
\end{variableblock}

  \begin{block}{Отношение сходства}
    \begin{equation*}\label{rhoSim}
      e_1\mathit{R}e_2 \Leftrightarrow \rho(e_1, e_2) \le \epsilon_0 \in \varepsilon(0)
    \end{equation*}
  \end{block}

\begin{variableblock}{Утверждение}{bg=SpringGreen,fg=black}{bg=SpringGreen,fg=red} 
\begin{center}
  Отношение сходства, заданное формулой \ref{rhoSim} является транзитивным.
\end{center}
\end{variableblock}

%% \begin{variableblock}{Следствие}{bg=SpringGreen,fg=black}{bg=SpringGreen,fg=red} 
%% \begin{center}
%%   Формализуя методы коллаборативных РС во введенной модели гарантируется выполнение необходимого и достаточного условия эффективности решения.
%% \end{center}
%% \end{variableblock}
\end{frame}

\begin{frame}
  \frametitle{Соотношение характеристик пользователей и объектов}
  \begin{variableblock}{Оценка сходства характеристик}{bg=SpringGreen,fg=black}{bg=SpringGreen,fg=red}
    $\delta_c : X \times Y \Rightarrow [0,1]$
  \end{variableblock}

  \begin{equation}
    \Psi :  U \rightarrow I
  \end{equation}


  \begin{equation}
    \nu_{\Psi}(y) = \underset{x \in X} {\mathrm{\sup}} \min\{ \delta_c(x,y); \mu(x) \}
  \end{equation}
  
  \begin{equation}
    \Psi(u) = \{ (y | \nu_{\Psi}(y)) \}, y \in Y
  \end{equation}
\end{frame}

\begin{frame}
  \frametitle{Соотношение характеристик пользователей и объектов. Пример}
\tiny{
\begin{block}{Дано:}
  \begin{itemize}
  \item $X = \{x_1=$ Наука, $x_2=$ Искусство $\}$, 
  \item $Y = \{y_1=$ Каллиграфия, $y_2=$ Химия, $y_3=$Фольклор $\}$
  \item
    \begin{tabular}{|c|c|c|c|c|}
      \hline
    & $y_1$   & $y_2$ & $y_3$\\ \hline
    $x_1$  &    0,3   & 1      & 0,5  \\ \hline
    $x_2$  &    1     & 0,3    & 0,5  \\ \hline
    \end{tabular}
  \item $u^a = \{ (c_1 | 1); (c_2 | 0,5) \}$
  \end{itemize}
\end{block}

\begin{block}{Решение:}
\begin{itemize}
\item $\nu(y_1) = \underset{x \in X} {\mathrm{\sup}}$ $\{ \min(0,3;1); \min(1;0,5)   \}$ = 0,5
\item $\nu(y_2) = \underset{x \in X} {\mathrm{\sup}}$ $\{ \min(1;1);   \min(0,3;0,5) \}$ = 1
\item $\nu(y_3) = \underset{x \in X} {\mathrm{\sup}}$ $\{ \min(0,5;1); \min(0,5;0,5) \}$ = 0,5
\item $\Psi(a) = \{ (y_1 | 0,5); (y_2 | 1); (y_3 | 0,5) \}$
\end{itemize}
\end{block}
}
\end{frame}

\begin{frame}
  \frametitle{Отношение сходства пользователя и объекта}
\begin{variableblock}{Расстояние между пользователем и объектом}{bg=SpringGreen,fg=black}{bg=SpringGreen,fg=red} 
\begin{center}
  $\rho(u,i) = \rho(\Psi(u), i)$
\end{center}
\end{variableblock}

\begin{variableblock}{Оценка сходства пользователя и объекта}{bg=SpringGreen,fg=black}{bg=SpringGreen,fg=red} 
\begin{center}
  $\delta_0(u,i) = 1 - \rho(\Psi(u), i)$
\end{center}
\end{variableblock}

\begin{variableblock}{Отношение сходства пользователя и объекта}{bg=SpringGreen,fg=black}{bg=SpringGreen,fg=red} 
\begin{center}
$u \mathit{R} i \Leftrightarrow$ $\rho(u,i) \le \epsilon_0$
\end{center}
\end{variableblock}
\end{frame}

\begin{frame}
  \frametitle{Оценка эффективности. Предпосылка}
  \begin{variableblock}{}{bg=SpringGreen,fg=black}{bg=SpringGreen,fg=red}
    \begin{center}
      Реальный пользователь: $a = \{(i^1, \rho(a, i^1)), (i^2, \rho(a, i^2)), ..., (i^n, \rho(a, i^n))\}$ \\
      Прототип пользователя: $a_1 = \{(i^1, \rho_1(a_1, i^1)), (i^2, \rho_1(a_1, i^2)), ..., (i^n, \rho_1(a_1, i^n))\}$ \\
      Тестовое множество: $\{ (i_{\times}, (1 - \rho(a, i_{\times})))  \}$, результирующее множество: $\{ (i_1, (1 - \rho(a, i_1)))  \}$ --- нечеткие множества.
    \end{center}
\end{variableblock}
\end{frame}

\begin{frame}
  \frametitle{Оценка эффективности качества работы РС}
\begin{variableblock}{Определение:}{bg=SpringGreen,fg=black}{bg=SpringGreen,fg=red}
\begin{center}
$\mathcal{E}(T^a_{1}, T^a_{\times}) = \frac{1}{|N|} \cdot \sum \limits_{n=1}^N |\rho(a, i_{\times}^n) - \rho_{1}(a, i_1^n)| $
\end{center}
\end{variableblock}

%% \begin{variableblock}{Для $\top$}{bg=SpringGreen,fg=black}{bg=SpringGreen,fg=red}
%% \begin{center}
%% $T^a_{1}, T^a_{\times}$ упорядочены по возрастанию $\rho_1(a, i_1)$ и $\rho(a, i_{\times})$ соответственно.
%% \end{center}
%% \end{variableblock}

\begin{variableblock}{}{bg=SpringGreen,fg=black}{bg=SpringGreen,fg=red}
  \begin{center}
    Является метрической функцией на множестве $T^a_1 \times T^a_{\times}$
  \end{center}
\end{variableblock}
\end{frame}

\begin{frame}
  \frametitle{Целевые оценки эффективности}
  \begin{columns}[T]
    \column{.5\textwidth} % Left column and width
    \begin{variableblock}{$\top$}{bg=SpringGreen,fg=black}{bg=blue,fg=white}
      $\mathcal{E}^a_{t} = 1 - \frac{1}{N} \cdot \sum \limits_{i_{1} \in T^a_1} s(n)$,\\$
      s(n) = 
      \begin{cases}
        1, &\text{$a  \mathit{R} i^n_{1}, n=1..N$}\\ %, $i^n_1 \in I_{1}^a, i^n \in I_{0}^a$}\\
        0, &\text{иначе}.
      \end{cases}
      $
    \end{variableblock}
    
    \column{.5\textwidth} % Left column and width
    \begin{variableblock}{Прогнозирование}{bg=SpringGreen,fg=black}{bg=blue,fg=white}
      $\mathcal{E}^a_{p} = \sum \limits_{i_{1} \in T^a_1} |\delta_0(a,i_{1}) - \delta_1(a,i_{1})|$
    \end{variableblock}
  \end{columns}
    
  \begin{variableblock}{}{bg=SpringGreen,fg=black}{bg=SpringGreen,fg=red}
    \begin{center}
       \textcolor{red}{$\mathcal{E}^a_t$ коррелирует с $\mathcal{E}^a_t$ и $\mathcal{E}^a_p.$}
    \end{center}
  \end{variableblock}
\end{frame}

\begin{frame}
  \frametitle{Решение задач РС}
  \begin{columns}[T]
    \column{.5\textwidth} % Left column and width
    \begin{variableblock}{}{bg=SpringGreen,fg=black}{bg=SpringGreen,fg=red}
      Линейный поиск схожих объектов $a \mathit{R} i$.
    \end{variableblock}

    \column{.5\textwidth} % Left column and width
    \begin{variableblock}{}{bg=SpringGreen,fg=black}{bg=SpringGreen,fg=red}
      $\delta_0(a, i_{\times}) = 1 - \rho_1(a, i_{\times})$
    \end{variableblock}
  \end{columns}

  \begin{variableblock}{Решения эффективны по критерию вычислительных затрат и $\mathcal{E}$}{bg=SpringGreen,fg=black}{bg=SpringGreen,fg=red}
  \end{variableblock}
\end{frame}

%%%%%%%%%%%%%%%%%%%%%%%%%%%%%
% chapter: CF
%%%%%%%%%%%%%%%%%%%%%%%%%%%%%%
\begin{frame}
  \frametitle{Классификация РС}
  \begin{itemize}
  \item Коллаборативные: 
    \begin{enumerate}
    \item Объектно-ориентированные, далее ОРС;
    \item Субъектно-ориентированные, далее СРС;
    \end{enumerate}
  \item Кластерные;
  \item Статистические: латентный анализ, факторный анализ, цепи Маркова, наивный Байесовский классификатор и т.д;
  \item Машинное обучение;
  \item Контентные;
  \end{itemize}
\end{frame}

%13
\begin{frame}
  \frametitle{ОРС}
  \begin{center}
    \includegraphics[width=3in,height=3in]{pics/ors-filter.pdf}
  \end{center}
\end{frame}

%14
\begin{frame}
  \frametitle{СРС}
  \begin{center}
    \includegraphics[width=4in,height=3in]{pics/srs-filter.pdf}
  \end{center}
\end{frame}

\begin{frame}
    \frametitle{Эвристика моделей Модели КРС\footnote{	
F. Ricci. Recommender Systems Handbook / F. Ricci, L. Rokach, B. Shapira, P. B. Kantor. // М.: Springer, 2011}}

    \begin{block}{\textcolor{blue}{Модели КРС основаны на эвристических утверждениях} }
    \begin{variableblock}{В общем виде РС определяет $\delta: U \times I \rightarrow [0,1]$}{bg=SpringGreen,fg=black}{bg=SpringGreen,fg=red}
      В ОРС и СРС $\delta_1: U \times I \rightarrow [0,1]$ \textcolor{red}{не задается}.
    \end{variableblock}
    \end{block}
\end{frame}

\begin{frame}
  \scriptsize{
    \frametitle{Эвристика схемы решения задачи $\top$}
    \begin{variableblock}{В общем виде РС определяет $\delta: U \times I \rightarrow [0,1]$}{bg=SpringGreen,fg=black}{bg=SpringGreen,fg=red}
      В ОРС и СРС $\delta_1: U \times I \rightarrow [0,1]$ \textcolor{red}{не задается}.
    \end{variableblock}
    
    \begin{columns}[T]
      %%%% \TOPN
      \column{.5\textwidth} % Left column and width
%% \hfill \break
%% \hfill \break
      \begin{variableblock}{Утверждение ОРС 1}{bg=SpringGreen,fg=black}{bg=SpringGreen,fg=red}
        \begin{equation}
          a \mathit{R} i_0 \wedge i_0 \mathit{R} i_{\times} \Rightarrow a \mathit{R} i_{\times}
          \label{ors1}
        \end{equation}
      \end{variableblock}
      
        %% \begin{variableblock}{Вид обучающего множества}{bg=SpringGreen,fg=black}{bg=SpringGreen,fg=red}
        %%   $T^a_0 = \{ i : a \mathit{R} i \}$
        %% \end{variableblock}

        %----------------------------------------------------------------
      %%%% PREDICT
        \column{.6\textwidth} % Left column and width
      \begin{variableblock}{Определение схожих пользователей}{bg=SpringGreen,fg=black}{bg=SpringGreen,fg=red}
        $a\mathit{R}u \Leftrightarrow$ $\forall i: \delta_0(a, i) \ne \varnothing \wedge \delta_0(u, i) \ne \varnothing:$ \\
        $|\delta_0(a, i) - \delta_0(u, i)| \le \epsilon_0$
      \end{variableblock}

      \begin{variableblock}{Утверждение СРС}{bg=SpringGreen,fg=black}{bg=SpringGreen,fg=red}
          \begin{equation}
              a\mathit{R}u \text{ для } T^a_0 \Rightarrow a\mathit{R}u \text{ для } T^a_{\times}
          \end{equation}
        \end{variableblock}
          %% \begin{variableblock}{Вид обучающего множества}{bg=SpringGreen,fg=black}{bg=SpringGreen,fg=red}
          %%   $T^a_0 = \{ (i, \delta_0(a, i) ) : \delta_0(a, i) \ne \varnothing \}$
          %% \end{variableblock}
    \end{columns}
  }
\end{frame}

% %%%%%%%%%%%%%%%%%%%%%%%
% Решение top-N
% %%%%%%%%%%%%%%%%%%%%%%%

\begin{frame}
  \scriptsize{
    \frametitle{Схема решения задачи $\top$}
    \begin{columns}[T]
      %%%% \TOPN
      \column{.5\textwidth} % Left column and width
      \begin{block}{\textcolor{blue}{ОРС. Кластер соседей $\mathcal{N}^0$ объектов множества $T^0$ }}
        \includegraphics[width=2in,height=1.9in]{pics/ors-solve-topn.pdf}
      \end{block}
        \begin{variableblock}{Вид обучающего множества}{bg=SpringGreen,fg=black}{bg=SpringGreen,fg=red}
          $T^a_0 = \{ (i, \delta_0(a, i)),  i : a \mathit{R} i \}$
        \end{variableblock}
        %----------------------------------------------------------------
      %%%% PREDICT
      \column{.6\textwidth} % Left column and width
      \begin{block}{\textcolor{blue}{СРС. Кластер соседей активного пользователя $\mathcal{N}^a$} }
        \includegraphics[width=2in,height=1.9in]{pics/srs-solve-topn.pdf}
      \end{block}
        \begin{variableblock}{Вид обучающего множества}{bg=SpringGreen,fg=black}{bg=SpringGreen,fg=red}
          $T^a_0 = \{ (i, \delta_0(a, i) ) : \delta_0(a, i) \ne \varnothing \}$
        \end{variableblock}
    \end{columns}
}
\end{frame}

\begin{frame}
  \scriptsize{
    \frametitle{Эвристика схемы решения задачи $\top$}
    \begin{variableblock}{В общем виде РС определяет $\delta: U \times I \rightarrow [0,1]$}{bg=SpringGreen,fg=black}{bg=SpringGreen,fg=red}
      В ОРС и СРС $\delta_1: U \times I \rightarrow [0,1]$ \textcolor{red}{не задается}.
    \end{variableblock}
    
    \begin{columns}[T]
      %%%% \TOPN
      \column{.5\textwidth} % Left column and width
      %% \hfill \break
      %% \hfill \break
      \begin{variableblock}{Утверждение ОРС 1}{bg=SpringGreen,fg=black}{bg=SpringGreen,fg=red}
        \begin{equation}
          a \mathit{R} i_0 \wedge i_0 \mathit{R} i_{\times} \Rightarrow a \mathit{R} i_{\times}
          \label{ors1}
        \end{equation}
      \end{variableblock}
      
      %% \begin{variableblock}{Вид обучающего множества}{bg=SpringGreen,fg=black}{bg=SpringGreen,fg=red}
      %%   $T^a_0 = \{ i : a \mathit{R} i \}$
      %% \end{variableblock}
      
      %----------------------------------------------------------------
      %%%% PREDICT
      \column{.6\textwidth} % Left column and width
      \begin{variableblock}{Утверждение СРС}{bg=SpringGreen,fg=black}{bg=SpringGreen,fg=red}
        \begin{equation}
          a\mathit{R}u \text{ для } T^a_0 \Rightarrow a\mathit{R}u \text{ для } T^a_{\times}
        \end{equation}
      \end{variableblock}
      %% \begin{variableblock}{Вид обучающего множества}{bg=SpringGreen,fg=black}{bg=SpringGreen,fg=red}
      %%   $T^a_0 = \{ (i, \delta_0(a, i) ) : \delta_0(a, i) \ne \varnothing \}$
      %% \end{variableblock}
    \end{columns}
      
    \begin{columns}[T]
      %%%% \TOPN
      \column{.5\textwidth} % Left column and width
      \begin{variableblock}{Решение $\top$ в ОРС}{bg=YellowGreen,fg=black}{bg=YellowGreen,fg=blue} 
        \begin{center}
          $T^a_1$ --- множество соседей мощности $N$
        \end{center}
      \end{variableblock}
      %----------------------------------------------------------------
      %%%% PBLUEICT
      \column{.6\textwidth} % Left column and width
      \begin{variableblock}{Решение $\top$ в СРС}{bg=YellowGreen,fg=black}{bg=YellowGreen,fg=blue} 
        \begin{center}
          $T^a_1 = \{ i : u \mathit{R} i, u \in \mathcal{N}^a \}$, $|T^a_1| = N$
        \end{center}
      \end{variableblock}
    \end{columns}
  }
\end{frame}

% %%%%%%%%%%%%%%%%%%%%%%%
% Решение прогнозирования
% %%%%%%%%%%%%%%%%%%%%%%%
\begin{frame}
  \scriptsize{
    \frametitle{Эвристика схемы решения задачи прогнозирования}
    \begin{variableblock}{В общем виде РС определяет $\delta: U \times I \rightarrow [0,1]$}{bg=SpringGreen,fg=black}{bg=SpringGreen,fg=red}
      В ОРС и СРС $\delta_1: U \times I \rightarrow [0,1]$ \textcolor{red}{не задается}.
    \end{variableblock}

    \begin{columns}[T]
      %%%% \TOPN
      \column{.6\textwidth} % Left column and width
        \begin{variableblock}{Утверждение ОРС 2}{bg=SpringGreen,fg=black}{bg=SpringGreen,fg=red}
          \begin{equation}
            a \mathit{R} i_0 \wedge i_0 \mathit{R} i_{\times} \Rightarrow |\delta_0(a, i_0) - \delta_0(a, i_{\times})| \le \epsilon_0
          \end{equation}
        \end{variableblock}
        
        %% \begin{variableblock}{Вид обучающего множества}{bg=SpringGreen,fg=black}{bg=SpringGreen,fg=red}
        %%   $T^a_0 = \{ i : a \mathit{R} i \}$
        %% \end{variableblock}
        %----------------------------------------------------------------
      %%%% PREDICT
      \column{.5\textwidth} % Left column and width
        \begin{variableblock}{Утверждение СРС}{bg=SpringGreen,fg=black}{bg=SpringGreen,fg=red}
          \begin{equation}
            a\mathit{R}u \text{ на } T^a_0 \Rightarrow a\mathit{R}u \text{ на } T^a_{\times}
          \end{equation}
        \end{variableblock}
        %% \begin{variableblock}{Вид обучающего множества}{bg=SpringGreen,fg=black}{bg=SpringGreen,fg=red}
        %%   $T^a_0 = \{ (i, \delta_0(a, i) ) : \delta_0(a, i) \ne \varnothing \}$
        %% \end{variableblock}
    \end{columns}
}
\end{frame}

\begin{frame}
  \scriptsize{
    \frametitle{Схема решения задачи прогнозирования оценки сходства $\delta_0(a, i_{\times})$}
    \begin{columns}[T]
      %%%% \TOPN
      \column{.5\textwidth} % Left column and width
      \begin{block}{\textcolor{blue}{ОРС. Кластер соседей $\mathcal{N}^{\times}$ прогнозируемого объекта $i_{\times}$} }
        \includegraphics[width=2in,height=1.5in]{pics/ors-solve-p.pdf}
      \end{block}
        \begin{variableblock}{Вид обучающего множества}{bg=SpringGreen,fg=black}{bg=SpringGreen,fg=red}
          $T^a_0 = \{ (i, \delta_0(a, i))\}$
        \end{variableblock}
        %----------------------------------------------------------------
      %%%% PREDICT
      \column{.6\textwidth} % Left column and width
      \begin{block}{\textcolor{blue}{СРС. Кластер соседей активного пользователя $\mathcal{N}^a$} }
        \includegraphics[width=2in,height=1.5in]{pics/srs-solve-topn.pdf}
      \end{block}

        \begin{variableblock}{Дополнительное условие}{bg=SpringGreen,fg=black}{bg=SpringGreen,fg=red}
        \begin{center}
          $u \in \mathcal{N}^a$, если $\exists$ $\delta_0(u, i_{\times})$
        \end{center}
      \end{variableblock}
    \end{columns}
}
\end{frame}

\begin{frame}
  \scriptsize{
    \frametitle{Решение задачи прогнозирования}
    \begin{variableblock}{В общем виде РС определяет $\delta: U \times I \rightarrow [0,1]$}{bg=SpringGreen,fg=black}{bg=SpringGreen,fg=red}
      В ОРС и СРС $\delta_1: U \times I \rightarrow [0,1]$ \textcolor{red}{не задается}.
    \end{variableblock}

    \begin{columns}[T]
      %%%% \TOPN
      \column{.6\textwidth} % Left column and width
        \begin{variableblock}{Утверждение ОРС 2}{bg=SpringGreen,fg=black}{bg=SpringGreen,fg=red}
          \begin{equation}
            a \mathit{R} i_0 \wedge i_0 \mathit{R} i_{\times} \Rightarrow |\delta_0(a, i_0) - \delta_0(a, i_{\times})| \le \epsilon_0
          \end{equation}
        \end{variableblock}
        
        %% \begin{variableblock}{Вид обучающего множества}{bg=SpringGreen,fg=black}{bg=SpringGreen,fg=red}
        %%   $T^a_0 = \{ i : a \mathit{R} i \}$
        %% \end{variableblock}
        %----------------------------------------------------------------
      %%%% PREDICT
      \column{.5\textwidth} % Left column and width
        \begin{variableblock}{Утверждение СРС}{bg=SpringGreen,fg=black}{bg=SpringGreen,fg=red}
          \begin{equation}
            a\mathit{R}u \text{ на } T^a_0 \Rightarrow a\mathit{R}u \text{ на } T^a_{\times}
          \end{equation}
        \end{variableblock}
        %% \begin{variableblock}{Вид обучающего множества}{bg=SpringGreen,fg=black}{bg=SpringGreen,fg=red}
        %%   $T^a_0 = \{ (i, \delta_0(a, i) ) : \delta_0(a, i) \ne \varnothing \}$
        %% \end{variableblock}
    \end{columns}

    \begin{columns}[T]
      %%%% \TOPN
      \column{.5\textwidth} % Left column and width
      \begin{variableblock}{Решение задачи прогнозирования в ОРС}{bg=YellowGreen,fg=black}{bg=YellowGreen,fg=blue}
        \begin{center}
          $\delta_1(a, i_{\times}) = p(\{ \delta_0(a, i), i \in \mathcal{N}^{\times} \})$ 
        \end{center}
      \end{variableblock}

      \begin{variableblock}{Решение задачи прогнозирования в СРС}{bg=YellowGreen,fg=black}{bg=YellowGreen,fg=blue}
        \begin{equation}\label{fpredict-ors}
          \frac{\sum \limits_{i \in \mathcal{N}^{\times}} \delta_{1}(i,i_{\times}) \cdot \delta_0(a,i)}{
            \sum \limits_{i \in \mathcal{N}^{\times}} \delta_{1}(i,i_{\times})
          }
        \end{equation}
      \end{variableblock}
      %----------------------------------------------------------------
      %%%% PREDICT
      \column{.6\textwidth} % Left column and width
      \begin{variableblock}{Прогнозная оценка}{bg=YellowGreen,fg=black}{bg=YellowGreen,fg=blue}
        \begin{center}
          $\delta_1(a, i_{\times}) = p(\{ \delta_0(p, i_{\times}), u \in \mathcal{N}^a \})$ 
        \end{center}
      \end{variableblock}

      \begin{variableblock}{Функция прогнозирования}{bg=YellowGreen,fg=black}{bg=YellowGreen,fg=blue}
        \begin{equation}
          \overline{\delta^a} + 
          \frac{ \sum \limits_{i \in \mathcal{N}^a} \delta_{1}(a, u^k) \cdot (\delta_0(u,i_{\times}) - \overline{\delta^k}) } 
               { \sum \limits_{u \in \mathcal{N}^a} | \delta_{1}(a, u^k) | } 
        \end{equation}
        $\overline{\delta^a}$ --- средняя оценка активного пользователя, $\overline{\delta^k}$ --- средняя оценка пользователя $u^k$. 

      \end{variableblock}
    \end{columns}
  }
\end{frame}

%%%%%%%%%%%%%%%%%%%
% Оценки эффективности
%%%%%%%%%%%%%%%%%%%
\begin{frame}
  \frametitle{Оценки эффективности $\mathcal{E}_p$ задачи прогнозирования}
    \begin{variableblock}{Целевая оценка эффективности}{bg=SpringGreen,fg=black}{bg=blue,fg=white}
      $\mathcal{E}^a_{p} = \sum \limits_{i_{1} \in T^a_1} |\delta_0(a,i_{1}) - \delta_1(a,i_{1})|$
    \end{variableblock}

  \begin{variableblock}{Используемые оценки}{bg=SpringGreen,fg=black}{bg=SpringGreen,fg=red}
    \begin{itemize}
    \item $MAE = \frac{1}{M} \cdot \sum \limits_{j=1}^M |\delta_1(a, i^j_{\times})  - \delta_0(a, i^j_{\times})|$
    \item $RMSE = \sqrt{\frac{1}{M} \cdot \sum \limits_{j=1}^M (\delta_1(a, i^j_{\times})  - \delta_0(a, i^j_{\times}))^2}$
    \item $M$ --- число проведенных прогнозов.
    \end{itemize}
  \end{variableblock}
\end{frame}

\begin{frame}
  \frametitle{Целевая оценка эффективности задачи $\top$}
  \begin{variableblock}{$\top$}{bg=SpringGreen,fg=black}{bg=blue,fg=white}
    $\mathcal{E}^a_{t} = 1 - \frac{1}{N} \cdot \sum \limits_{i_{1} \in T^a_1} s(n)$,\\$
    s(n) = 
    \begin{cases}
      1, &\text{$a  \mathit{R} i^n_{1}, n=1..N$}\\ %, $i^n_1 \in I_{1}^a, i^n \in I_{0}^a$}\\
      0, &\text{иначе}.
    \end{cases}
    $
  \end{variableblock}
\end{frame}


\begin{frame}
  \frametitle{Объектно-ориентированная оценки эффективности $\mathcal{E}^i_{t}$}
  \begin{variableblock}{В общем виде РС определяет $\delta: U \times I \rightarrow [0,1]$}{bg=SpringGreen,fg=black}{bg=SpringGreen,fg=red}
    В ОРС и СРС $\delta_1: U \times I \rightarrow [0,1]$ \textcolor{red}{не задается}.
  \end{variableblock}

  \begin{variableblock}{Основаны на утверждении ОРС}{bg=SpringGreen,fg=black}{bg=SpringGreen,fg=red}
        \begin{center}
          $a  \mathit{R} i_{\times} \wedge i_1 \mathit{R} i_{\times} \Rightarrow a \mathit{R} i_1$
        \end{center}
      \end{variableblock}

  $
  s(n) = 
  \begin{cases}
    1, &\text{$\exists$ $i_{\times} \in T^a_{\times}$: $i^n_{1} \mathit{R} i_{\times}, n=1..N$}\\ %, $i^n_1 \in I_{1}^a, i^n \in I_{0}^a$}\\
    0, &\text{иначе}.
  \end{cases}
  $
\end{frame}

\begin{frame}
  \frametitle{Используемые оценки эффективности задачи $\top$}
  \begin{itemize}
  \item Точность: $1 - \frac{1}{N} \cdot \sum \limits_{n=1}^N s(i_1^n)$;
  \item Точность по списку длины $L$: $P@L = 1 - \frac{1}{L} \cdot \sum \limits_{n=1}^L s(n)$.
  \item $AveP: \frac{1}{\sum \limits_{n=1}^{N} s(n)} \cdot \sum \limits_{L=1}^{N} P@L$.
  \item $NDCG: 1 - \frac{DCG}{IDCG}$, где $DCG = s(1) + \sum \limits_{n=2}^N \frac{s(n)}{log_2(n)}$.
    $IDCG$ --- идеальное $DCG$, для которого $\forall$ $n=1..N$ $r(n) = 1$.
  \end{itemize}
\end{frame}

%% \begin{frame}
%% %ToDo график где по y оценка, по x - K
%%   \frametitle{Обобщенная оценка эффективности решения задачи $\top$}
%%   \begin{variableblock}{}{bg=SpringGreen,fg=black}{bg=SpringGreen,fg=red}
%%     \begin{center}
%%       Целевая оценка эффективности задачи $top-N$ --- \\ функция $\mathcal{E}^a_{t}(K)$
%%     \end{center}
%%   \end{variableblock}

%%   \begin{variableblock}{}{bg=SpringGreen,fg=black}{bg=SpringGreen,fg=red}
%%     \begin{center}
%%       Объектно-ориентированная оценка эффективности задачи $top-N$ --- \\ функция $\mathcal{E}^i_{t}(K)$
%%     \end{center}
%%   \end{variableblock}

%%   \begin{variableblock}{}{bg=SpringGreen,fg=black}{bg=SpringGreen,fg=red}
%%     \begin{center}
%%       Metrics within each equivalency class were strongly correlated, while metrics from different equivalency
%%       classes were uncorrelated.\footnote{
%%         J. L. Herlocker. Evaluating collaborative filtering recommender systems. / 
%%         J. L. Herlocker, J. A. Konstan, L. G. Terveen, J. Riedl // ACM Trans. Inf. Syst Т 22(1), pp 5-53, 2004}
%%     \end{center}
%%   \end{variableblock}

%% \end{frame}

\begin{frame}
  \frametitle{Корреляция $\mathcal{E}^i_{t}$ и $\mathcal{E}^a_{t}$}

    %% \begin{equation}\label{use-e0-1}
    %%   \mathcal{E}^i_{t} < \epsilon_1 \Rightarrow \mathcal{E}^a_{t} < \epsilon_1
    %% \end{equation}
    %%     \end{variableblock}

    %%     \begin{variableblock}{}{bg=SpringGreen,fg=black}{bg=SpringGreen,fg=red}
    %%       \begin{equation}\label{use-e0-2}
    %%         \mathcal{E}^a_{t} < \epsilon_1 \Rightarrow \mathcal{E}^i_{t} < \epsilon_1 
    %%       \end{equation}
    %%     \end{variableblock}
  \begin{variableblock}{}{bg=SpringGreen,fg=black}{bg=SpringGreen,fg=red}
    \begin{center}
      Значения $\mathcal{E}^i_{t}$ объективны, если $\mathcal{E}^i_{t}$ коррелирует $\mathcal{E}^a_{t}$
    \end{center}
  \end{variableblock}

  \begin{variableblock}{Утверждение о применимости $\mathcal{E}^i_{t}$}{bg=SpringGreen,fg=black}{bg=SpringGreen,fg=red}
    \begin{center}
      Необходимым и достаточным условием применения оценки эффективности $\mathcal{E}^i_{t}$ является выполнение утверждения 
      ОРС 1:\\
      $a \mathit{R} i_0 \wedge i_0 \mathit{R} i_{\times} \Rightarrow a \mathit{R} i_{\times}$
    \end{center}
  \end{variableblock}
\end{frame}

\begin{frame}
  \frametitle{История действий пользователя $T^a_0$.}
      \begin{columns}[T]
      %%%% \TOPN
      \column{.6\textwidth} % Left column and width
      \begin{variableblock}{Утверждение ОРС 1}{bg=SpringGreen,fg=black}{bg=SpringGreen,fg=red}
        \begin{equation*}
          a \mathit{R} i_0 \wedge i_0 \mathit{R} i_{\times} \Rightarrow a \mathit{R} i_{\times}
          \label{ors1}
        \end{equation*}
      \end{variableblock}

      \begin{variableblock}{Утверждение ОРС 2}{bg=SpringGreen,fg=black}{bg=SpringGreen,fg=red}
        \begin{equation*}
          a \mathit{R} i_0 \wedge i_0 \mathit{R} i_{\times} \Rightarrow |\delta_0(a, i_0) - \delta_0(a, i_{\times})| \le \epsilon_0
        \end{equation*}
      \end{variableblock}
      
      \begin{variableblock}{Вид обучающего множества}{bg=SpringGreen,fg=black}{bg=SpringGreen,fg=red}
        $T^a_1 = \{ (i_1, 1),  i_1 : i_0 \mathit{R} i_1, i_0 \in T^a_0 \}$
      \end{variableblock}

  
        %% \begin{variableblock}{Вид обучающего множества}{bg=SpringGreen,fg=black}{bg=SpringGreen,fg=red}
        %%   $T^a_0 = \{ i : a \mathit{R} i \}$
        %% \end{variableblock}
        %----------------------------------------------------------------
      %%%% PREDICT
      \column{.5\textwidth} % Left column and width
        \begin{variableblock}{Утверждение СРС}{bg=SpringGreen,fg=black}{bg=SpringGreen,fg=red}
          \begin{equation*}
            a\mathit{R}u \text{ на } T^a_0 \Rightarrow a\mathit{R}u \text{ на } T^a_{\times}
          \end{equation*}
        \end{variableblock}
        %% \begin{variableblock}{Вид обучающего множества}{bg=SpringGreen,fg=black}{bg=SpringGreen,fg=red}
        %%   $T^a_0 = \{ (i, \delta_0(a, i) ) : \delta_0(a, i) \ne \varnothing \}$
        %% \end{variableblock}
    \end{columns}

 \end{frame}

\begin{frame}
  \frametitle{Свойства входных данных}
  \begin{block}{}
    \begin{center}
      \underline{Данные меняются во времени}
    \end{center}
  \end{block}

  \begin{columns}[T]
    \column{.5\textwidth} % Left column and width
    \begin{variableblock}{Влияние на ОРС}{bg=SpringGreen,fg=black}{bg=SpringGreen,fg=blue}
      $a \mathcal{R} i_0 \wedge i_0 \mathcal{R} i_{\times}, \delta(a, i_{\times}) \le \Delta$
    \end{variableblock}
    \column{.5\textwidth} % Left column and width
    
    \begin{variableblock}{Влияние на СРС}{bg=SpringGreen,fg=black}{bg=SpringGreen,fg=blue}
      $a\mathit{R}u \text{ для } T^a_0$ $\wedge$ $\delta_0(a, u) \le \Delta$ для $T^a_{\times}$
    \end{variableblock}
  \end{columns}

  \begin{block}{}
    \begin{center}
      \underline{Предпочтения пользователя неоднородны}
    \end{center}
  \end{block}

  \begin{columns}[T]
    \column{.5\textwidth} % Left column and width
    \begin{variableblock}{Влияние на ОРС}{bg=SpringGreen,fg=black}{bg=SpringGreen,fg=blue}
      $\exists$ $i_0,i_{\times}:$ $a \mathcal{R} i_0 \wedge a \mathcal{R} i_{\times}$, $\delta_0(i_0, i_{\times}) \le \Delta$.
    \end{variableblock}
    \column{.5\textwidth} % Left column and width
    \begin{variableblock}{Влияние на СРС}{bg=SpringGreen,fg=black}{bg=SpringGreen,fg=blue}
      $a\mathit{R}u^1$ $\wedge$ $a\mathit{R}u^2$, при этом $\delta_1(u^1, u^2) \le \Delta$ 
      % $a\mathit{R}u \text{ для } T^1_0$ $\wedge$ $\delta_0(a, u) \le \Delta$ для $T^2_0$, где $T^a_0 = T^1_0 \bigcup T^2_0$
    \end{variableblock}
  \end{columns}
\end{frame}

\begin{frame}
  \frametitle{Свойства входных данных. Динамика}
    \begin{variableblock}{Данные меняются во времени. Последствия для ОРС}{bg=SpringGreen,fg=black}{bg=SpringGreen,fg=red}
      $a \mathcal{R} i_0 \wedge i_0 \mathcal{R} i_{\times}, \delta(a, i_{\times}) \le \Delta$
    \end{variableblock}

    \begin{variableblock}{Данные меняются во времени. Последствия для СРС}{bg=SpringGreen,fg=black}{bg=SpringGreen,fg=red}
      $a\mathit{R}u \text{ для } T^a_0$ $\wedge$ $\delta_0(a, u) \le \Delta$ для $T^a_{\times}$
    \end{variableblock}
 \end{frame}

\begin{frame}
  \frametitle{Свойства входных данных. Неоднородность предпочтений}
    \begin{variableblock}{Предпочтения пользователя неоднородны. Последствия для ОРС}{bg=SpringGreen,fg=black}{bg=SpringGreen,fg=red}
      $\exists$ $i_0,i_{\times}:$ $a \mathcal{R} i_0 \wedge a \mathcal{R} i_{\times}$, $\delta_0(i_0, i_{\times}) \le \Delta$.
    \end{variableblock}

    \begin{variableblock}{Предпочтения пользователя неоднородны. Последствия для СРС}{bg=SpringGreen,fg=black}{bg=SpringGreen,fg=red}
      $a\mathit{R}u^1$ $\wedge$ $a\mathit{R}u^2$, при этом $\delta_1(u^1, u^2) \le \Delta$ 
      % $a\mathit{R}u \text{ для } T^1_0$ $\wedge$ $\delta_0(a, u) \le \Delta$ для $T^2_0$, где $T^a_0 = T^1_0 \bigcup T^2_0$
    \end{variableblock}
 \end{frame}

\begin{frame}
  \frametitle{Свойства входных данных. Влияние}
  \begin{variableblock}{Утверждение о применимости $\mathcal{E}^i_{t}$}{bg=SpringGreen,fg=black}{bg=SpringGreen,fg=red}
    \begin{center}
      Необходимым и достаточным условием применения оценки эффективности $\mathcal{E}^i_{t}$ является выполнение утверждения 
      ОРС 1:\\
      $a \mathit{R} i_0 \wedge i_0 \mathit{R} i_{\times} \Rightarrow a \mathit{R} i_{\times}$
    \end{center}
  \end{variableblock}

  \begin{variableblock}{}{bg=SpringGreen,fg=red}{bg=blue,fg=red}
    \begin{itemize}
    \item Эвристические утверждения могут не выполниться: потенциальное ухудшение эффективности решения.
    \item Нарушение объективного условия применения оценки эффективности $\mathcal{E}^i_t$.
    \end{itemize}
  \end{variableblock}
\end{frame}

%%%%%%%%%%%%%%%%%%%%%%%%%%%%%%%%%%%%%%%%%%%
% Необходимое и достаточное условие эффективности
%%%%%%%%%%%%%%%%%%%%%%%%%%%%%%%%%%%%%%%%%%%
\begin{frame}
  \scriptsize{
    \frametitle{Схема решения задачи $\top$}
    \begin{columns}[T]
      %%%% \TOPN
      \column{.5\textwidth} % Left column and width
      \begin{block}{\textcolor{blue}{ОРС. Кластер соседей $\mathcal{N}^0$ объектов множества $T^0$ }}
        \includegraphics[width=2in,height=1.9in]{pics/ors-solve-topn.pdf}
      \end{block}
        \begin{variableblock}{Вид обучающего множества}{bg=SpringGreen,fg=black}{bg=SpringGreen,fg=red}
          $T^a_0 = \{ (i, \delta_0(a, i)),  i : a \mathit{R} i \}$
        \end{variableblock}
        %----------------------------------------------------------------
      %%%% PREDICT
      \column{.6\textwidth} % Left column and width
      \begin{block}{\textcolor{blue}{СРС. Кластер соседей активного пользователя $\mathcal{N}^a$} }
        \includegraphics[width=2in,height=1.9in]{pics/srs-solve-topn.pdf}
      \end{block}
        \begin{variableblock}{Вид обучающего множества}{bg=SpringGreen,fg=black}{bg=SpringGreen,fg=red}
          $T^a_0 = \{ (i, \delta_0(a, i) ) : \delta_0(a, i) \ne \varnothing \}$
        \end{variableblock}
    \end{columns}
}
\end{frame}

\begin{frame}
  \frametitle{Условия эффективности решений задачи $\top$}
  \begin{variableblock}{}{bg=SpringGreen,fg=black}{bg=SpringGreen,fg=red}
    Необходимым и достаточным условием эффективности решения задачи $\top$ является выполнение транзитивности отношения сходства 
    объектов на множестве $T^a_0 \bigcup T^a_1 \bigcup T^a_{\times}$:
    \begin{equation}
      i_0 \mathit{R} i_{1} \wedge i_0 \mathit{R} i_{\times}  \Rightarrow i_{1} \mathit{R} i_{\times}
    \label{transAssert1}
    \end{equation}
  \end{variableblock}
\end{frame}

\begin{frame}
  \scriptsize{
    \frametitle{Схема решения задачи прогнозирования оценки сходства $\delta_0(a, i_{\times})$}
    \begin{columns}[T]
      %%%% \TOPN
      \column{.5\textwidth} % Left column and width
      \begin{block}{\textcolor{blue}{ОРС. Кластер соседей $\mathcal{N}^{\times}$ прогнозируемого объекта $i_{\times}$} }
        \includegraphics[width=2in,height=1.5in]{pics/ors-solve-p.pdf}
      \end{block}
        \begin{variableblock}{Вид обучающего множества}{bg=SpringGreen,fg=black}{bg=SpringGreen,fg=red}
          $T^a_0 = \{ (i, \delta_0(a, i))\}$
        \end{variableblock}
        %----------------------------------------------------------------
      %%%% PREDICT
      \column{.6\textwidth} % Left column and width
      \begin{block}{\textcolor{blue}{СРС. Кластер соседей активного пользователя $\mathcal{N}^a$} }
        \includegraphics[width=2in,height=1.5in]{pics/srs-solve-topn.pdf}
      \end{block}

        \begin{variableblock}{Дополнительное условие}{bg=SpringGreen,fg=black}{bg=SpringGreen,fg=red}
        \begin{center}
          $u \in \mathcal{N}^a$, если $\exists$ $\delta_0(u, i_{\times})$
        \end{center}
      \end{variableblock}
    \end{columns}
}
\end{frame}

\begin{frame}
  \frametitle{Условия эффективности решений задачи прогнозирования}
    \begin{columns}[T]
      %%%% \TOPN
      \column{.5\textwidth} % Left column and width
      \begin{variableblock}{ОРС}{bg=SpringGreen,fg=black}{bg=SpringGreen,fg=red}
        %Выполнение транзитивности отношения сходства объектов на кластере соседей $\mathcal{N}^{\times}$ прогнозирвуемого объекта
        \begin{assert}
          Достаточным условием эффективности решения прогнозирования является выполнение транзитивности отношения сходства объектов на множестве $\mathcal{N}^{\times}$:
          \begin{equation}\label{suff-cond-pred-ors}
            i_{\times} \mathit{R} i^1 \wedge i_{\times} \mathit{R} i^2  \Rightarrow i^1 \mathit{R} i^2, i^m \in \mathcal{N}^{\times}
          \end{equation}
        \end{assert}
      \end{variableblock}
        %----------------------------------------------------------------
      %%%% PREDICT
      \column{.6\textwidth} % Left column and width
        \begin{variableblock}{СРС}{bg=SpringGreen,fg=black}{bg=SpringGreen,fg=red}
            %Выполнение транзитивности отношения сходства между ползователями кластера соседей $\mathcal{N}^a$ активного пользователя
            \begin{assert}
              Необходимым и достаточным условием эффективности решения задачи прогнозирования является выполнение транзитивности отношения сходства 
              пользователей на множестве соседей активного пользователя $\mathcal{N}$:
              \begin{equation}\label{nec-cond-pred-srs}
                a \mathit{R} u^1 \wedge a \mathit{R} u^2  \Rightarrow u^1 \mathit{R} u^2, u^m \in \mathcal{N}
              \end{equation}
            \end{assert}
        \end{variableblock}
    \end{columns}
\end{frame}

\begin{frame}
  \frametitle{Выполнимость условий эффективности решений}
  \begin{variableblock}{Зависит $\delta_1$ и $\Delta$}{bg=SpringGreen,fg=black}{bg=SpringGreen,fg=red}
    \begin{center}
      Традиционные функции, используемые в качестве $\delta_1$: косинус и коэффициент корреляции Пирсона.
    \end{center}
  \end{variableblock}

    \begin{variableblock}{Транзитивность при использовании косинуса}{bg=SpringGreen,fg=black}{bg=SpringGreen,fg=red}
    \begin{center}
      $\Delta = 0.49$ $\wedge$ $\delta_0(i^1_0, i^2_0) = 0.5$ $\wedge$ $\delta_0(i^1_0, i_{\times}) = 0.5 \cancel{\Rightarrow} \delta_0(i^2_0, i_{\times}) > \Delta$.
    \end{center}
  \end{variableblock}

  \begin{variableblock}{Транзитивность при использовании коэффициента корреляции Пирсона}{bg=SpringGreen,fg=black}{bg=SpringGreen,fg=red}
    \begin{center}
      Оценка сходства, использующая коэффициент корреляции Пирсона не обладает свойством транзитивности 
\footnote{A. Castro Sotos, S. Vanhoof, W. Van den Noortgate, P. Onghena.  The non-transitivity of Pearson's correlation coefficient: an educational perspective. // 
Proceedings of the 56th Session of the ISI / Т 62, C 4609-4613}.
    \end{center}
  \end{variableblock}

  \begin{variableblock}{}{bg=SpringGreen,fg=red}{bg=blue,fg=red}
    \begin{center}
      Эффективное решение для любых ОРС и СРС не гарантируется.
    \end{center}
  \end{variableblock}
\end{frame}

\begin{frame}
  \frametitle{Асимптотическая сложность алгоритмов\footnote{G. Linden, B. Smith and J. York. Amazon.com recommendations: 
item-to-item collaborative filtering. // Internet Computing 7:1 / С 76-80, 2003}}
    \begin{columns}[T]
      %%%% \TOPN
      \column{.5\textwidth} % Left column and width
      \begin{variableblock}{ОРС}{bg=SpringGreen,fg=black}{bg=SpringGreen,fg=red}
        Для решения строится матрица $\mathcal{M}^m_n = \delta_1(i^m, i^n)$. Асимптотическая сложность $O(|I|^2)$.
      \end{variableblock}

      \column{.6\textwidth} % Left column and width
      \begin{variableblock}{СРС}{bg=SpringGreen,fg=black}{bg=SpringGreen,fg=red}
        Решение основано на линейном поиске соседей. Асимптотическая сложность $O(|U|)$.
      \end{variableblock}
    \end{columns}
\end{frame}

\begin{frame}
  \frametitle{Решение задач РС}
  \begin{columns}[T]
    \column{.5\textwidth} % Left column and width
    \begin{variableblock}{}{bg=SpringGreen,fg=black}{bg=SpringGreen,fg=red}
      Линейный поиск схожих объектов.
    \end{variableblock}

    \column{.5\textwidth} % Left column and width
    \begin{variableblock}{}{bg=SpringGreen,fg=black}{bg=SpringGreen,fg=red}
      $\delta_0(a, i_{\times}) = 1 - \rho_1(a, i_{\times})$
    \end{variableblock}
  \end{columns}
\end{frame}

\begin{frame}
  \frametitle{Сравнение асимптотических сложностей решений}
  \begin{center}
    \begin{table}[h]
      \begin{tabular}{|c|c|c|c|}
        \hline	
        Модель & Задача & Сложность  \\ \hline
        ОРС & $\top$ & $O(|T|^2)$  \\ \hline
        ОРС & Прогнозирование & $O(|T|^2)$ \\ \hline
        СРС & $\top$ & $O(|U|)$ \\ \hline
        СРС & Прогнозирование & $O(|U|)$ \\ \hline
        Нечеткая & $\top$ & $O(|T|)$ \\ \hline
        Нечеткая & Прогнозирование & $O(C)$ \\ \hline
      \end{tabular}
    \end{table}
    \end{center}
\end{frame}

\begin{frame}
  \frametitle{Выводы}
  \begin{itemize}
    \item Коллаборативные модели не гарантируют эффективность по критерию устойчивости при динамике и неоднородности данных.
    \item Коллаборативные модели для люых параметров $\delta_1$ и $\Delta$ не гарантируют эффективный результат.
    \item В общем случае оценка эффективности $\mathcal{E}^i_t$ может быть необъективной.
    \item Коллаборативные модели не обладают высокой масштабируемостью. 
  \end{itemize}
\end{frame}

\begin{frame}
  \frametitle{Отношение сходства во введенной модели. Свойства}
\begin{variableblock}{$\rho(e_1, e_2)$ обладает метрическими свойствами\footnote{
А. Кофман. Введение в теорию нечетких множеств // М.: Радио и связь, 1982. 432 c}
}{bg=SpringGreen,fg=black}{bg=SpringGreen,fg=red}  
\end{variableblock}

\begin{variableblock}{Утверждение}{bg=SpringGreen,fg=black}{bg=SpringGreen,fg=red} 
\begin{center}
  Отношение сходства, заданное формулой \ref{rhoSim} является транзитивным.
\end{center}
\end{variableblock}

\begin{variableblock}{Следствие}{bg=SpringGreen,fg=black}{bg=SpringGreen,fg=red} 
\begin{center}
  Формализуя методы коллаборативных РС во введенной модели, гарантируется выполнение необходимого и достаточного условия эффективности решения.
\end{center}
\end{variableblock}
\end{frame}

\begin{frame}
  \frametitle{Модификация алгоритма составления множества соседей}
  \begin{variableblock}{}{bg=SpringGreen,fg=black}{bg=SpringGreen,fg=red}
    \begin{equation}
      e_1\mathit{R}e_2 \Leftrightarrow \rho(e_1, e_2) \le \epsilon_0 \in \varepsilon(0)
    \end{equation}
  \end{variableblock}

  \begin{variableblock}{Дополнительное условие добавления элемента в множество соседей $\mathcal{N}$}{bg=SpringGreen,fg=black}{bg=SpringGreen,fg=red}
    \begin{equation*}\label{method}
      e \mathit{R} c \wedge e \mathit{R} e^{\bigcup} \wedge e \mathit{R} e^{\bigcap}, 
    \end{equation*}
  \end{variableblock}

    \begin{itemize}
      \item $c$ --- центр множества соседей.
      \item $c^{\bigcup} = e^1 \bigcup e^2 \bigcup ... e^{M} \bigcup e$, $e^j \in \mathcal{N}, j=1..M$.
      \item $c^{\bigcap} = e^1 \bigcap e^2 \bigcap ... e^{M} \bigcap e$, $e^j \in \mathcal{N}, j=1..M$.  
    \end{itemize}

    \begin{assert}
      Введенная модификация гарантирует выполнение необходимых и достаточных условий, накладываемых на отношение сходства в кластерах соседей.
    \end{assert}
\end{frame}

\begin{frame}
  \frametitle{Корреляция $\mathcal{E}$ и $\mathcal{E}^i_t$}
  \begin{variableblock}{Утверждение о применимости $\mathcal{E}^i_{t}$}{bg=SpringGreen,fg=black}{bg=SpringGreen,fg=red}
    \begin{center}
      Необходимым и достаточным условием применения оценки эффективности $\mathcal{E}^i_{t}$ является выполнение утверждения 
      ОРС 1:\\
      $a \mathit{R} i_0 \wedge i_0 \mathit{R} i_{\times} \Rightarrow a \mathit{R} i_{\times}$
    \end{center}
  \end{variableblock}
  
  \begin{variableblock}{Утверждение}{bg=SpringGreen,fg=black}{bg=SpringGreen,fg=red}
    \begin{center}
      Оценка с $\mathcal{E}$ коррелирует с $\mathcal{E}^i_t$, если выполняется условие применимости $\mathcal{E}^i_t$.
    \end{center}
  \end{variableblock}

  \begin{variableblock}{Утверждение}{bg=SpringGreen,fg=black}{bg=SpringGreen,fg=red}
    \begin{center}
      Оценка с $\mathcal{E}$ может использоваться для различного класса задач.
    \end{center}
  \end{variableblock}

  \begin{variableblock}{}{bg=SpringGreen,fg=black}{bg=SpringGreen,fg=red}
    \begin{center}
      Применение $\mathcal{E}$ не обусловлено дополнительными ограничениями такими, как свойства входных данных.
    \end{center}
  \end{variableblock}
\end{frame}
  
%% \begin{frame}
%%   \frametitle{Выводы. Плюсы и минусы}
%% \end{frame}

\begin{frame}
  \frametitle{Множество данных MovieLens}
  \begin{itemize}
    \item $|U|$ = 6000
    \item $|T|$ = 10000 --- множество фильмов
    \item $| \{ \delta_0(u^m, i^n) \} |$ = 1 000 000
    \item $|Y| = 18$ --- множество кинематографических жанров; 
    \item $X$ --- оценки пользователей;
    \item $\nu_l(y_i) = \{0,1\}$
    \item Существующие данные удовлетворяют КРС: $u = \{ (i, \delta_0(u, i)) \}$
  \end{itemize}
\end{frame}

\begin{frame}
  \frametitle{Составление контента пользователя для множества Movie Lens в нечеткой модели}
  \begin{block}{}
    Предполагается, что между оценкой пользователя и жанрами, характерными для объекта существует корреляция.
  \end{block}


  \begin{block}{$\nu(y_k) = \underset{x \in X} {\mathrm{\sup}} \min\{ \delta_c(x,y); \mu(x) \}, \delta_c : T_0 \times Y \rightarrow [0,1]$} 
    $\nu(y_k) = \underset{i \in T_0} {\mathrm{\sup}} \min\{ \delta_c(i,y_k) : \nu_i(y_k) = 1; \delta_0(u, i) \}$ \\

    $\delta_c(i, y_k) = |\{ i : \delta_0(u, i) > \Delta, \nu(y_k) = 1\}| - |\{ i : \delta_0(u, i) < \Delta_1, \nu(y_k) = 1 \}|$,
    $\Delta = 0.8$, $\Delta_1 = 0.6$
  \end{block}
\end{frame}

\begin{frame}
  \frametitle{Сравнение коллаборативных моделей и коллаборативных моеделей, формализоавнных в нечеткой модели}
\scriptsize{
    \begin{columns}[T]
      \column{.6\textwidth} % Left column and width
      \begin{variableblock}{Задача $\top$. ОРС}{bg=SpringGreen,fg=black}{bg=SpringGreen,fg=red}
        \begin{table}[h] 
          \begin{center}
            \begin{tabular}{|c|c|c|c|c|}
              \hline	
              Модель & P & AveP & NDCG  \\ \hline
              ОРС & 0.0105 & 0.011  & 0.0419  \\ \hline
              Нечеткая ОРС  & 0.0004 &  0.0004 & 0.0318  \\ \hline
            \end{tabular}
          \end{center}
        \end{table}
      \end{variableblock}

      \column{.5\textwidth} % Left column and width
      \begin{variableblock}{Задача $\top$. СРС}{bg=SpringGreen,fg=black}{bg=SpringGreen,fg=red}
          \begin{table}[h] 
            \begin{center}
              \label{testResult:predict-trans-ors}
            \begin{tabular}{|c|c|c|c|c|}
              \hline	
              Модель & P & AveP & NDCG  \\ \hline
              СРС & 0.0185 & 0.018  & 0.0392  \\ \hline
              Нечеткая СРС & 0.0057 &  0.007 & 0.0368  \\ \hline
            \end{tabular}
            \end{center}
          \end{table}
      \end{variableblock}
    \end{columns}
      %----------------------------------------------------------------
      %%%% PREDICT
    \begin{columns}[T]
      \column{.6\textwidth} % Left column and width
      \begin{variableblock}{Задача прогнозирования. ОРС}{bg=SpringGreen,fg=black}{bg=SpringGreen,fg=red}
        \begin{table}[h] 
          \begin{center}
            \begin{tabular}{|c|c|c|}
              \hline	
              Модель      & MAE & RMSE \\ \hline
              ОРС         &  0.184 & 0.225  \\ \hline
              Нечеткая ОРС &  0.178 & 0.22  \\ \hline
            \end{tabular}
        \end{center}
        \end{table}
      \end{variableblock}
            
      \column{.5\textwidth} % Left column and width
      \begin{variableblock}{Задача прогнозирования. СРС}{bg=SpringGreen,fg=black}{bg=SpringGreen,fg=red}
        \begin{table}[h] 
          \begin{center}
            \begin{tabular}{|c|c|c|}
              \hline	
              Модель      & MAE & RMSE \\ \hline
              Традиционный & 0.168 & 0.222  \\ \hline
              Нечеткая СРС & 0.155  & 0.213  \\ \hline
            \end{tabular}
          \end{center}
        \end{table}
        \end{variableblock}
      \end{columns}

      \begin{variableblock}{}{bg=SpringGreen,fg=black}{bg=SpringGreen,fg=red}
        \begin{table}[h] 
          \begin{center}
            $e^1 \mathit{R} e^2 \Leftrightarrow \rho_1(e^1,e^2) < 0.1$\\
            $e^1 \mathit{R} e^2 \Leftrightarrow \delta_1(e^1,e^2) > 0.9$
          \end{center}
        \end{table}
        \end{variableblock}
}
\end{frame}


\begin{frame}
  \frametitle{Эффективность применения модифицированного алгоритма построения кластера соседей}
  \scriptsize{
      \begin{variableblock}{Задача $\top$. СРС}{bg=SpringGreen,fg=black}{bg=SpringGreen,fg=red}
        \begin{table}[h] 
          \begin{center}
            \begin{tabular}{|c|c|c|c|}
              \hline	
              Алгоритм & P & AveP & NDCG  \\ \hline
              Традиционный & 0.0185 & 0.018  & 0.0392  \\ \hline
              Модифицированный & 0.0171 & 0.017  & 0.0382 \\ \hline
            \end{tabular}
          \end{center}
        \end{table}
      \end{variableblock}

    \begin{columns}[T]
      %%%% \TOPN
      \column{.6\textwidth} % Left column and width

      \begin{variableblock}{Задача прогнозирования. ОРС}{bg=SpringGreen,fg=black}{bg=SpringGreen,fg=red}
          \begin{table}[h] 
            \begin{center}
              \begin{tabular}{|c|c|c|}
                \hline	
                Тип & MAE & RMSE \\ \hline
                Традиционный ОРС & 0.184 & 0.225  \\ \hline
                Модифицированный & 0.182  & 0.220  \\ \hline
              \end{tabular}
            \end{center}
          \end{table}
      \end{variableblock}
      %----------------------------------------------------------------
      %%%% PREDICT
      \column{.6\textwidth} % Left column and width
        \begin{variableblock}{Задача прогнозирования. СРС}{bg=SpringGreen,fg=black}{bg=SpringGreen,fg=red}
          \begin{table}[h] 
        \begin{center}
          \begin{tabular}{|c|c|c|}
            \hline	
            Тип & MAE & RMSE \\ \hline
            Традиционный & 0.168 & 0.222  \\ \hline
            Модифицированный & 0.162 & 0.214 \\ \hline
          \end{tabular}
        \end{center}
      \end{table}
      \end{variableblock}
    \end{columns}
      \begin{variableblock}{}{bg=SpringGreen,fg=black}{bg=SpringGreen,fg=red}
        \begin{table}[h] 
          \begin{center}
            $e^1 \mathit{R} e^2 \Leftrightarrow \rho_1(e^1,e^2) < 0.5$\\
            $e^1 \mathit{R} e^2 \Leftrightarrow \delta_1(e^1,e^2) > 0.9$
          \end{center}
        \end{table}
        \end{variableblock}
}
\end{frame}



\begin{frame}
  \frametitle{Сравнение КРС и нечеткой модели. Задача $\top$}
  \begin{variableblock}{}{bg=SpringGreen,fg=black}{bg=SpringGreen,fg=red}
    \begin{table}[h] 
      \begin{center}
        \begin{tabular}{|c|c|c|c|c|}
          \hline	
          Условия выполняются/&&&&\\Модель & P & AveP & NDCG & $\mathcal{E}$ \\ \hline
          Да/ОРС ($cos$) & 0.0105 & 0.011  & 0.0419 & 0.57 \\ \hline
          Да/Нечеткая &&&&\\модель & 0.0472 & 0.053 & 0.0693 & 0.079 \\ \hline
          Нет/ОРС ($cos$) & 0.5619 & 0.5986  & 0.5921 & 0.7114 \\ \hline
          Нет/Нечеткая &&&&\\модель & 0.0472 & 0.053 & 0.0693 & 0.079 \\ \hline
          Случайное &&&&\\ решение  & 0.7141& 0.7994 & 0.7541 & 0.5632 \\ \hline
        \end{tabular}
      \end{center}
    \end{table}
  \end{variableblock}
\end{frame}

\begin{frame}
  \frametitle{Сравнение КРС и нечеткой модели. Задача прогнозирования}
  \begin{variableblock}{}{bg=SpringGreen,fg=black}{bg=SpringGreen,fg=red}
    \begin{table}[h] 
      \begin{center}
        \begin{tabular}{|c|c|c|}
          \hline	
          Условие выполняется/Модель & MAE & RMSE \\ \hline
          Да/СРС ($cos$) & 0.164 & 0.215  \\ \hline
          Да/Нечеткая модель & 0.183 & 0.247  \\ \hline
          Нет/СРС (Пирсон) &  0.191  & 0.25  \\ \hline
          Нет/Нечеткая модель & 0.183 & 0.247  \\ \hline
          Случайное решение  & 0.253 & 0.31 \\ \hline
        \end{tabular}
      \end{center}
    \end{table}
  \end{variableblock}
\end{frame}

\begin{frame}
  \frametitle{Веб-интерфейс пользователя разработанной кинематографической РС. Выбор кинематографических жанров}
  \begin{center}
    \includegraphics[width=3in,height=2.5in]{pics/ml-interface.png}
  \end{center}
\end{frame}

\begin{frame}
  \frametitle{Веб-интерфейс кинематографической РС. Результат запроса и интерфейс работы с ним}
  \begin{center}
    \includegraphics[width=3in,height=2.5in]{pics/ml-rslt.png}
  \end{center}
\end{frame}

\begin{frame}
  \frametitle{Веб-интерфейс пользователя разработанной музыкальной РС. Выбор музыкальных жанров}
  \begin{center}
    \includegraphics[width=4.5in,height=2.5in]{pics/lastfm-interface.png}
  \end{center}
\end{frame}

\begin{frame}
  \frametitle{Веб-интерфейс музыкальной РС. Результат запроса и интерфейс работы с ним}
  \begin{center}
    \includegraphics[width=4.5in,height=2.5in]{pics/lastfm-rslt.png}
  \end{center}
\end{frame}

\begin{frame}
  \frametitle{Выводы}
  \scriptsize{
  \begin{itemize}
  \item Проведен анализ коллаборативных момделей в ходе которого выявлены:
    \begin{itemize}
    \item Необходимое и достаточные условия эффективности решений основных задач в РС и условия их выполнения.
    \item Необходимые и достаточные условия применимости существующих оценок эффективности решения задачи $\top$ и условия их выполнения.
    \end{itemize}
  \item Разработанная математическая модель РС, в которой:
        \begin{itemize}
        \item Формализованы коллаборативные модели, за счет чего повышена их эффективность.
        \item Введена цценка эффективности во введенной модели, обладающая объективностью результата оценки и коррелирующая с существующими оценками эффективности.
        \item Релизованы алгоритмы решения задач во введенной модели, обладающие асимптотической сложностью, меньшей на порядок, по сравнению 
          с существующими решениями и которые теоретически эффективны по введенной оценке эффективности.
        \end{itemize}
  \item Разработано ПО для проведения тестирования. В ходе тестирования были получены результаты на реальных данных, подтверждающие теоретические результаты.
  \item Разработано ПО, реализующее введенную нечеткую модель.
  \end{itemize}
}
\end{frame}

\begin{frame}
  \frametitle{Опубликованные статьи}
  \scriptsize{
Опубликованные:
\begin{enumerate}
\item Понизовкин Д. М. Оптимальное распределение проектов при проведении экспертизы / Д. М. Понизовкин, С. А. Амелькин // 
Электронные библиотеки: Перспективные Методы и Технологии, Электронные коллекции. --- 2010. --- С. 524-525.
\item Понизовкин Д. М. Построение оптимального графа связей в системах коллаборативной фильтрации / Д. М. Понизовкин, С. А. Амелькин // 
Программные системы: теория и приложения. 2011.--- Т. 2. --- № 4. С. 107–114
\item Понизовкин Д. М. Математическая модель коллаборативных процессов принятия решений / Д. М. Понизовкин, С. А. Амелькин // 
Программные системы: теория и приложения. 2011. --- Т. 2. --- № 4. С 95-99.
\item Амелькин С. А. Оптимальное проведение экспертизы образовательных процессов / С. А. Амелькин, Д. М. Понизовкин // 
Труды XVII Всероссийской научно-методической конференции Телематика’2010, Санкт-Петербург: Университетские телекоммуникации. --- 2010. ---
№ 1, С. 158-159.
\item Д. М. Понизовкин. Влияние меры сходства на результативность РС // Программные системы: теория и приложения, 2014. --- т. 2. --- N. 5. С 55–65.
\end{enumerate}
}
\end{frame}

\begin{frame}
  \frametitle{Ожидаемые статьи}
  \scriptsize{
    \begin{enumerate}
    \item Понизовкин Д. М. Повышение эффективности решения задачи прогнозирования в эвристических коллаборативных рекомендательных системах / Д. М. Понизовкин // 
      Искусственный Интеллект и Принятие Решений.
    \item Понизовкин Д. М. Повышение эффективности решения задачи top-N в эвристических коллаборативных рекомендательных системах / Д. М. Понизовкин // 
      Информационные технологии
    \item Понизовкин Д. М. Применение точности как оценки эффективности решения задачи top-N в объектных коллаборативных рекомендательных система / 
      Д. М. Понизовкин, С. А. Амелькин // Программная инженерия
    \item Понизовкин Д. М. Модели рекомендательных систем, основанные на теории нечетких множеств / Д. М. Понизовкин // Искусственный Интеллект и Принятие Решений.
    \end{enumerate}
  }
\end{frame}

\begin{frame}
  \begin{block}{}
    \begin{center}
      {\LARGE Спасибо за внимание!}
      \end{center}
  \end{block}
\end{frame}

%%%%%%% END
\end{document}
