\documentclass[a4paper,14pt]{proc}
%\documentclass[14pt]{article}
%\documentclass[a4paper,14pt]{extarticle}
\renewcommand{\baselinestretch}{1.5}
\renewcommand{\figurename}{Рисунок} 
\usepackage{ntheorem}

\usepackage[T2A]{fontenc}
\usepackage[utf8]{inputenc}
\usepackage[english,russian]{babel}
\usepackage{graphicx}
\usepackage{array}
\usepackage{tabularx}
\usepackage{setspace}

\usepackage{color}
\usepackage{url}
\usepackage{multicol}
\usepackage{amssymb}
\usepackage{hyperref}
%\usepackage[linesnumbered,boxed]{algorithm2e}
\usepackage{algorithm}
\usepackage{algpseudocode}

\usepackage{longtable} 
\usepackage{amsmath}
\usepackage{enumerate}
\begin{document}
\scriptsize{
\begin{enumerate}

%
% ====================== INTRO ===================================
%

% 2
\item Вступление
\item Перед человеком постоянно встает задача предпочтительного выбора.
С учетом огромного числа доступной информации эта задача ставновится трудной и актуальной.
Рекомендательные системы являются одним из инструментов решения задачи предпочтительного 
выбора или связяанной с ней задачи определения степени предпочтения пользователя конкретному выбору.
Качественная РС та, которая решает задачи эффективно по некольким критериям: насколько решение
соответствует реальным предпочтениям пользователя, вычислительные затраты, устойчивость результата
при изменении входных данных. 

Очень часто такие системы встречаются в интернет-магазинах, так как являются
инструментом маркетинга со стороны владельца и инструментом возможной экономии времени и денег 
со стороны пользователей. Такие системы встречаются там, где перед пользователем стоит задача выбора: 
что посмотреть, почитать, послушать. Справа вы найдете малую толику названий работающих РС.

%3
\item Задача выбора называется задачей топ-н. Задача определения степени предпочтения - задачей прогнозирования.

%4
\item Существуют различные виды моделей РС. Основные модели приведены на слайде. 
%5
\item Основные задачи исследования --- определить математическую модель РС, 
в которой решения будут эффективны по критериям масштабируемости, реальным предпочтениям и устойчивостию
Ввести сопсоб оценке эффективности по критерию предпочтений пользователя, обладающий объективностью.
Провести тестирование введенной моделти на реальных данных и на тех
же данных провести тестирование одной из существующих распространненых моделях. Сравнить результаты
%6
\item Рассматриваемые РС опреруют с двумся множествами: множеством польззователей и множеством объектов
некоторой прикладной облсасти. Пользователи и объекты являются элементами данных.
Выше говорилось о неформальном термине <<предпочтение>> и его степени. Введем термин <<сходство>>,
который синонимичен понятию релевантность, используемому в информационном поиске. Чем выше оценка сходства, 
тем степень предпочтения пользователя объектом выше. В общем виде функциональность РС заключается в определении 
значения оценки сходства. Оценка сходства может быть задана самим пользователем, 
либо быть вычислена системой. Помимо оценки сходства пользователя и объекта РС могут рассчитывать оценку сходства объектов или
пользователей с соответствующей областью определения.

%7
\item Определение оценки сходства производится по известным метаданным о пользователях и объектах,
которве будем называть контентом. Единицу метаданных будем называть характеристикой.
Будем считать, что между контентом и элементом данных существует биективное отображение.
Говоря об элементе, будет так же иметь в виду и его контент. 

Будем говорить, что между элементами выполняется отношение сходства,
если оценка сходства больше некоторого порогового дельта. Такие элементы будем называть схожими или соседями. 
Это отношение является рефлексивным и симметричным. А выполнение свойства транзитивности является одним из 
ключевых моментов работы и будет рассмотрено дальше.

Для выявления эффективности решения используются функции называемые оценками эффективности, которые
рассчитываются от декартова произведения тестового и результирующего множеств/. Чем меньше значение этой
функции, тем эффективность выше.

% 8
\item Задаим формально задачи:
\begin{enumerate}
\item top-n: определить подмножество объектов мощности $N$ схожих с пользователем. 
\item прогнозирование: вычислить с минимальной погрешностью оценку сходства активного пользователя и прогнозируемого объекта 
\end{enumerate}
А целью РС заключатеся в том, чтобы оценка эффективности на этих решениях стремилась к минимуму.

% 9
\item Тогда целевые оценки эффективности, то есть как бы оценил результат пользователь, будут следующими:
\begin{enumerate}
\item top-n: определить отношение числа объектов результирующей выборки схожих с активным пользователем к числу н
\item прогнозирование: определить погрешность
\end{enumerate}

%
% =================== FUZZY ============================
%

% 10
\item Будем представлять контенты элементов системы в виде нечетких подмножеств множеств характесистик.
Определим операции объединения и пересечения контентов.

% 10
\item Между элементами системы зададим функцию расстояния в виде обощенного расстояния Хэмминга, через расстояние
определим оценку сходства и отношение сходства.

%12
\item В описываемой модели, введенное расстояние обладает метрическими свойствами. Поэтому при определнии отношения сходства
через расстояние выполняется свойство транзитивности этого отношения. 

%13
\item Характеристики пользователей и объектов сопоставимы. Для некоторых систем сопоставление может быть тождественным. К примеру, характеристики объектов ---  
музыкальные жанры, в которых написано музыкальное произведение, являющееся объектом. Характеристики пользователя --- музыкальные жанров, которые он предпочитает. Однако 
возможны и более сложные сопоставления. К примеру, для системы банка, где объекты --- акции, где характеристики --- финансовые категории, а пользователи ---
клиенты банка, где характеристики  социально-демографиескими показатели.

Для реализации сопоставлдения нужно ввести нечеткое отображения дельта ц. Будем называть его
оценкой сходства характеристик. Оценка сходства характеристик может быть задана аналитически или таблично с помощью эксперта, 
методов машинного обучения, классификации и т.д. Введя оценку сходства мы можем задать отображение пользователей на множество объектов.

%14
\item Приведем небольшой надуманный пример для пояснения. Оценка сходства характеристик задана экспертом в виде следующей таблицы.
Используя формулы, введенные на предыдущем слайде, получим следующий контент объекта, являющийся образом контента пользователя.

%15
\item Введя отображение пользователя на множество объектов, можно задать расстояние, оценку сходства и
отношение сходства пользователя и объекта. Таким образом, введенная модель удовлетворяет целевой функциональности РС, которая 
заключается в определении оценки сходства пользователя и объекта.  

%16
\item В процессе решения задачи РС, по сути, формирует прототипа пользователя. Рассматрия процесс решения с такой стороны такой точки зрения, 
определим эффективность по предпочтениям пользователя как расстояние между прототипом и реальным пользователем.
Для этого задаим результирующее и тестовое множества во введенной модели как нечеткие подмножества универсального множества объектов.
И впользуемся введенным ранее обощенным рассятоянием Хэмминга. Введенная оценка эффективности является метрикой, а значит может служить
количественным показателем эффективности.

%18
\item Напомним вид целевых оценок.
Введенная оценка эффективности коррелирует с целевыми оценками эффективности решения задачи $top-N$ и прогнозирования,
а, значит, ее значения являются объективным показателем эффективности решения. 

\item Введя расстояние между пользователем и объектом, задачу топ-н можно решить простым линейным поиском
по множеству объектов, а задача прогнозирования заключается только в расчете расстояния. Теоретически, решения
оптимальны по критерию введенной оценки эффективности, а, значит, эффективны по предпочтениям. Асимптотическая 
сложность решения не высока, то есть решения эффективны по критерию вычислительной сложности. Асимптотическая сложность
была вычислена при учете того, что атомарной операцией является вычисление расстояния.
 
%
% =================== CRS ====================
%
% 
% 20
\item Коллаборативные модели являются
распространненными и считаются хорошо изученными и называются в некоторых исследованиях традиционными.
Они успешно используются во многих коммерческих проектах. Проведем анализ этих моделей и сравним с введенной.

% 22
\item Идею коллаборативной фильтрации можно представить следующими двумя свхемами. Первая схема задает ОРС. 
Они отфильтровывают объекты, нехсожие с теми, что схожи с активным пользователем.
К примеру, для  системы интернет-магазина, схожесть между пользователем и объектом
определяется тем, какие товары купил пользователь, и система отфильтровывает те товары, что не схожи с ранее купленными.

%23
\item Идея СРС заключается в том, что отфильтровываются пользователи, которые не схожи с активным по его предпочтениям и вкусам. 
И информация, известная о схожих пользователяз используется для дальнейшего решения.

%24.
\item Коллаборативные модели не задают целевую оценку сходства между пользователем и объхектом, 
но выводят предположения о ее значении, исходя из известных данных и эвристических утверждений, 
которые заложены в их модель.

%25
\item При решении задачи $top-N$ в ОРС используется утверждение, которое гласит, что 
{\it пользователи обладают тенденцией давать высокую оценку сходства объектам, схожим с теми, которые были высоко оценены ранее}. 

Будем говорить, что пользователи схожи, если разница между оценками сходства, которые они задали одним и тем же объектам, меньше некоторого малого
эпсилон 0. 

В СРС используется утверждение, которое гласит, что {\it если пользователи схожи по данным обучающей выборки, то они будут схожи по данным тестовой выборки}.

%26
\item У решений в обеих моделях есть общая часть --- построение кластера соседей. 
Для ОРС центр кластера соседей --- это обучающее множество. 
В него входят объекты, схожие с активным пользователем. Для СРС центр кластера --- сам активный пользователь.

%27
\item Решение задачи $top-N$ в ОРС --- это сам построенный кластер мощности $N$. Для СРС --- это $N$ объектов, которые были высоко 
оценены соседями.

%28 
\item Эвристическое утверждение решения задачи прогнозирования в ОРС гласит, что  
{\it значения оценок сходства, поставленные пользователем объектам, между которыми выполняется отношение сходства, приблизительно равны}. 
Эвристическое утверждение решения задачи прогнозирования 
в СРС то же, что и при решении задачи $top-N$.

%29
\item При решении задачи прогнозирования в ОРС строится кластер соседей с центром, являющимся прогнозируемым объектом.
Для СРС --- строится кластер с центром, являющимся активным пользователем с наложением дополнительного условия на вхождение
пользователей в кластер: они должны были задать оценку сходства прогнозируемому объекту.

%30
\item Решением задачи прогнозирования для ОРС является значение прогнозной функции, взятой от оценок сходства, которые 
задал аквтиный пользователь соседям прогнозируемого объекта. Решение в СРС --- это значение функции прогнозировния, взятое
от значений оценок сходства, которые поставили соседи прогнозируемому объекту. 
Распространенные оценки прогнозирования --- средняя взвешенная оценка.

%31-33
\item Для существующих РС оценки эффективности разбиваются по классам задач. Оценки, которые принадлежат одному классу коррелируют 
между собой, поэтому будет рассматривать обощенные оценки эффективности. Для задачи прогнозирования используются функции, которые 
коррелируют с целевой оценкой эффективности.

Цель задачи $top-N$  --- предоставить такое множество, что его объекты схожи с активным
пользователем. Для определения примеров часто используемых оценок определим вспомогательную функцию с,
равную 1, если объект результирующего множества с порядковым номер н схож с активным пользователем. 0 --- иначе.
Результирующее множество упорядочено. 

%34.
\item Напомним вид целевой оценки эффективности задачи топ-н. Однако ОРС не вводят такой функции s(n).
Для ОРС оценка эффективности примет такой вид. Основываясь на эвристическом туверждении.

\item Однако, объективность применения объектно-ориентированной оценки эффективности решения задачи топ-н обусловлена ее корреляцией. В диссертации было
выведено необходимое и достаточное условие объективного применения объектно-ориентированной оценки, а именно тогда и только тогда, когда
утверждение ОРС 1. 

%36
\item В отличе от введенной модели, для коллаборативных систем важна история, то есть данные обучающего множества.
Эта история фигурирует и в эвристических утверждениях и в схемах решения.

При этом , если говорить о итории, то следует сказать, что данные, скоторыми имеют дело реальные 
РС динамичные. Помимо постоянного наполнения множеств данных с течением времени меняются нужды и предпочтения пользователей,
поэтому  могут нарушиться утверждения ОРС и СРС.

%37
\item Помимо этого, вкусы пользователей обладают свойством неоднородности. То есть пользователи могут предпочитать несхожие объекты.
Наличие неоднородности данных в обучающем множестве и в актуальном, то есть в тестовом, может нарушить выполнение 
эвристических утвержэдений.

%38
\item В диссертационном исследовании было показано, что невыполняние эваристических утверждений возмиожное 
в силу свойств реальных данных, является причиной потенциального снижения эффективности решения и невозможности 
применения оценки эффективности задачи $top-N$.

%39

\item Далее рассмотрим схемы решений. Напомни, что для решения задачи топ-н строится кластер соседей
с центром являющимяс обучающим множеством в случае ОРС. И кластер с центром активным пользователем для СРС.
Условия эффективности --- выполнение транзитивности на объединении тестового обучающего и резхультирующего множеств.
%40
\item Для решения задачи прогнозирования строим кластеры с центром прогнозируемый объект и активный пользовтель для ОРС и СРС соответственно.
Дасточное условие эффективного решения задачи прогнозирования в ОРС --- это выполнение свойства транзитивности отношения сходства на кластере
соседей прогнозирвуемого объекта. Условия эффективного  --- это выполнение свойства транзитивности на кластере соседей. 

%41
\item Выполнение транзитивности отношения сходства зависит от того, какая функция используется в качестве оценки сходства и значения параметра дельта. 
Для традицонных мер сходства таких как косинус и коэффициент корреляции Пирсона выполнение необходимых и достаточных условий эффективности
не гарантировано.
%% Традиционные функции для ОРС --- это косинус. Для СРС --- кожэффицициент корреляции Пирсона. Пусть в системе используется косинус и параметр дельта равен 0.49. И пусть есть три объекта
%% между 1 и второым и между вторым и третьим оценка сходства равна 0.5. Но из этого совсем не следует что значение оценки сходства
%% будет так же равно 0.5 и выполняится отношение сходства. 

%% Коэффициент корреляции Пирсона не обладет свойством транзитивности.

%% Таким оразом для любых параметров дельта 1 и дельта большое не гарантируется выполнение условий эффективности, а, значит,
%% и эффективного решения.

%42
\item Для ОРС характерно составление матрицы, которой хранятся значения оценок сходства.
При решении задачи прогнозирования проивходится линейный поиск соседей. 

\item Решение во введенной модели в случае задачи топ-н линейный поиск, в случае прогнозирования --- единичный рассчет расстояния.
Для приведенных решений в обеих моделях характерно одно и то же действие --- вычислдение значения сходства или расстояния. Возьмем
это действие за атомарную единицу и рассмаотрим асмптотическую сложность решений

Решения во введенной модели имеют сложность, на порядок меньшу по сравнению с коллаборативными моделями.

%43 
\item Таким образом, проведя анализ можно сделать следующий вывод, что, вобщем случае, коллаборативные системы,
хоть и считаются хорошо изученными, но имеют неустойчивость по отношению к эффективности и применнию оценки эффективности 
для задачи топ-н при работе с динамическими и неоднороднами данными и при различных функциях оценки сходства и параметра дельта.
Следствием чего являются исселдования, в которых анализируется не модель, а предлагаются различные оценки эффективности с различными 
весовыми коэффициентами.

\item Напомним, что мы ввели модель, в которой выполняется тарнзитивность отношения сходства, столь необходимая для эфыффективного
решения в коллаборативных моделях. Таким образом, формализовав коллаборативные системы в нечеткой модели можно повысить эффективность их решения.

\item Для реальных систем ввденное отношение сходства в нечеткой модели может быть строгим в силу малого значения
эпсилон нулевое. Для таких случаев 
реализована модификация, которая гарантирует выполнение неодходимых и достсачных условий транзитивности отношения 
сходства соседей на кластере для различных оценок сходства и любых дельта.

\item Введенная оценка эффективности коррелирует с объектно-ориентированной оценкой эффективности решения задачи топ-н, если
выполняется необходимое и достаточное условие применимости объектно-ориентированной оценки. Таким образом введенная оценка
коррелирует с целевыми оценками и оценками, которые принадлежат классам различных задач, а так же она не зависит от входных
данных. Поэтому может быть использована в качестве стандарта для любой системы, протестированной на любых данных.

%49
\item Далее приведем результаты тестов, которые были получены с помощью разработанного программного обеспечения.
Разработанное ПО реализует коллаборативные модели и нечеткую. С помощью ПО были решены задачи прогнозирования и топ-н
описанными ранее способыми. 
%% Данное программное решает задачи топ-н и прогнозирования в каждой модели и 
%% расчитывает оценки эффективности. На слайде описаны данныеЮ используемые для тестов. Это известное в кругу исследователей РС множество Movie Lens, собранное на реальныхз
%% данных группой исследователей РС. Во входном множестве находятся данные об оценках 6000 пользователей, всего оценок порядка миллиона,
%% 10 000 объеках. Множество характеристик объектов --- это множество кинематографических жанров мощности 18. Значение функции 
%% приналоедности контентов объектов принадлежат шкале 0,1. Елиница --- когда жанр характерен для фильма.
%50 
\item Вид оценки сходства характеристик основан на предположении, что между оценкой пользователя на объект и между
характеристиками объекта есть корреляция. Характеристическая функция принадлежности контента пользователя  вычислялась по следующей формуле.

Хочу здесь сделать ремарку о том, что для точного составления контента объекта, мощность характеристик слишком мала. На оценку пользователя
влияют гораздо большее число факторов, чем просто жанр. Объектам, которым характерны одни и теюбе жанры, пользователь может поставить 
совершенно различные оценки.

\item На слайде приведены стандартные решения задач в коллаобративных РС и те же решения после формализации коллаборативных моделей
в нечетких. Результаты подтверждают теоретический вывод об эффективности данной формализации

\item Здесь приведены результаты решения задач топ-н и прогнозирования при использовании значения порогового знечения 
В первом случае решение --- традиционное, во втором приведен модифицированный алгоритм. Результаты показывают, что использование модифицированного адгоритма
эффективно. Результаты на моде хуже, чем в модели, так как знижена точность эпсилон.

\item Были проведены тесты на множествах сформированных спецаильно так, что не выполняются необходимые и достаточные условия транзитивности.
В том числе задачи были решены слуайно. Если условия не выполняются, то решения в коллаборативных системах приближаются к случайным. 
Свойства данных не вляиют на эффективность решений в нечеткой модели и в среднем решения в этих показывают результат лучше, чем коллаборативные модели.

\item Был разработано ядро нечеткой модели для веб-приложений, которое позволяющет работать с любой базой данных определнной структуры. 
Была разработана версия веб-приложения, рботающая на кинематографическиой базе данных муви ленс. На слайде показан пример ее интерфейса, где
пользователь сам задает сво контет указывая те жанры, котонры ему интересны. 
\item Далее ядро находит в базе данных фильмы, схожие с заданным контентом. Реализовано тождественное отображение. 
\item Так же ядро было использовано для веб-приложения поиска музыкальных исполнителей быза данных last fm. Составление контента стало более расширенным.
Пользователь может задать вес инетерсующего жанра.  
\end{enumerate}
\end{document}
