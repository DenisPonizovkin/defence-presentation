\documentclass[a4paper,11pt]{proc}
%\documentclass[14pt]{article}
%\documentclass[a4paper,14pt]{extarticle}
\usepackage{ntheorem}

\usepackage[T2A]{fontenc}
\usepackage[utf8]{inputenc}
\usepackage[english,russian]{babel}
\usepackage{graphicx}
\usepackage{array}
\usepackage{tabularx}
\usepackage{setspace}

\usepackage{color}
\usepackage{url}
\usepackage{multicol}
\usepackage{amssymb}
\usepackage{hyperref}
%\usepackage[linesnumbered,boxed]{algorithm2e}

\usepackage{longtable}
\usepackage{amsmath}
\usepackage{enumerate}
\begin{document}
\scriptsize{
\begin{enumerate}

%
% ====================== INTRO ===================================
%

% 2
\item Добрый день, в докладе будет представлена разработанная в ходе диссертационного исследования
	математическая модель рекомендательной системы, основанная на теории нечетких
	множеств. Будет показана, что разработанная модель является эффективным расширением коллаборативной
	модели. Докладчик --- соискатель степени кандидата технических наук, Понизовкин Денис Михайлович.
	Научный руководитель --- кандидат технических наук Амелькин Сергей Анатольевич.

\item Одной из наиболее изученных, распространенных и успешных по отношению к коммерческой реализации
	моделей рекомендательных систем, является коллаборативная
	модель. Несмотря на это, существуют открытые
	проблемы, связанные с их применением: отсутствие общего формального
	описания --- в существующих исследованиях один и
	тот же подход может носить различные наименования;
	методы решений задач основаны на аксиомах, которые являются эвристическими утверждениях,
	достоверность которых не проанализирована;
	анализ качества применяемых подходов является
	эмпирическим, он производится на каких-то конкретных данных, поэтому
	качество одного и того же подхода вырьируется от набора данных;
	высокая асимптотическая сложность алгоритмов при применении их на реальном
	множестве исходных данных.

\item Цель исследования --- разработать математическую модель, которая
	позволяет применять методы коллаборативной фильтрации более эффективно,
	и в разработанной модели определить собственные эффективные методы решений
	для которых существующие проблемы коллаборативных методов аннулируются;

\item РС оперирует метаданными о пользователях и объектах. Единицу метаданных
	назовем характеристикой, структуру данных, содержащую метаданные, назовем
	контентом.

	Ключевой информацией РС является информация о предпочтениях пользователя.
	Как правило, такая информация представляется в виде матрицы, элементами
	которой являются значения функции оценки близости. Это значение показывает,
	насколько конкретный объект близок по своим характеристикам пользователю.
	Будем считать, что чем меньше значение, тем объект ближе. Будем говорить,
		что между объектом и пользователем выполняется отношение близости,
		если значение оценки близости мало.

	\item
		РС решают две основные задачи: (1) прогнозирование --- вычисление
	неизвестного значения оценки близости на конкретной паре пользователь, объект,
	которое осуществляется за счет алгоритмического вычисления прогнозной
	функции ро с чертой.

	(2) topn --- сформировать подмножество объектов мощности н,
	между которыми и активным пользователем выполняется отношение близости.

	Решение производится по обучающей выборки, далее, во время тестирования,
	полученный результат сравнивается с тестовым множеством.
	Стоит отметить, что для коллаборативных моделей, которые будут описаны
		далее, при решении задачи топн в обучающее и тестовое множества входят
		только те объекты, между которыми и активным пользователем выполняется
		отношение близости, остальная информация является избыточной.

	\item Модель РС --- это тройка, которая задает способ представления
		данных о пользователях и объектах и правило алгоритмического вывода
		значений прогнозной функции.

	\item Коллаборативная модель делится на два по типу тому принципу, какое
		множество фильтруется при решении задачи. Либо фильтруется множество
		объектов, и тогда такие модели называются объектно-ориентированными,
		либо фильтруется множество пользователей, и тогда такие модели
		называются субъектно-ориентированными. Как правило,
		объектно-ориентированные модели применяются для решения задачи топн.
		Субъектно-ориентированные --- для решения задачи прогнозирования.

		Правила вывода обоих типов коллаборативной модели основаны на
		аксиомах, которые представляют собой эвристические утверждения.

		Правило вывода объектно-ориентированной модели основано на
		эвристическом утверждении, которое гласит, что если пользователю
		нравится объект и, а объект и поход на объект жи, то пользователю
		понравится объект жи.
		Во введенной терминологии неформальные понятия
		нравится, похож заменим на отношение близости, где выполнение
		отношения близости объектов устанавливается на основании значений
		функций, именуемых мерой близости. Стандартной мерой таких моделей
		является косинус угла между векторами, которыми являются контенты
		объектов. Правило вывода
		объектно-ориентированной модели говорит о том, что если существует
		объект, между которым и объектом обучающего множества выполняется
		отношение близости, то между таким объектом и активным пользователем
		будет выполняться отношение близости. Соответственно, алгоритм решения
		задачи топн основан на поиске таких объектов, между которыми и
		объектами обучающего множества выполняется отношение близости.

		Правило вывода субъектно-ориентированной модели основано на
		эвристическом утверждении, которое гласит, что если между
		пользователи в прошлом были схожи по предпочтениям, то они будут
		сходи по предпочтениям и в будущем.
		Во введенной терминологии неформальное понятие
		схож заменим на отношение близости пользователей, факт выполнения
		которого устанавливается на основании значений
		функций, именуемых мерой близости. Стандартной мерой близости таких
		моделей --- коэффициент корреляции.
		Правило вывода субъектно-ориентированной модели говорит о том,
		что значение оценки близости активного пользователя можно функционально
		выразить через значения оценки близости пользователей, между которыми
		и активным выполняется отношение близости. Стандартный алгоритм решения
		задачи прогнозирования в субъектно-ориентированной модели заключается
		в двух итерациях, на первой строится
		так называемый кластер соседей, центром которого
		является активный пользователь, а элементами --- пользователи, между
		которыми и активным выполняется отношение близости. На второй итерации
		по оценкам близости элементов кластера вычисляется значение прогнозной
		функции. Для субъектно-ориентированных моделей характеристиками пользователей
		являются объекты, а значением веса --- значение близости.


	\item эффективностью назовем свойство системы удовлетворять некоторому
		критерию независимо от дополнительных условий или ограничений.
		Проанализируем эффективность коллаборативной модели.
		Рассмотрим три критерия, по которым будем анализировать эффективность:
		(1) качество получаемого решения, которое для задачи топн определяется
		отношением числа объектов результирующего множества, между
		которыми и активным пользователем выполняется близости, к числу н.
		Задача топн решена качества, если отношение больше либо равно
		некоторого заданного параметра.
		Показателем качества решения задачи прогнозирования является
		погрешность между значением прогнозной функции и реальным значением
		оценки близости. Задача прогнозирования решена качественно, если
		погрешность мала.
		(2) второй критерий --- это стабильность. Стабильностью наозовем
		формировать качественное решение независимо от свойств исходных данных.
		(3) третий критерий --- это вычислительная сложность, которую будем
		характеризовать асимптотической сложностью алгоритмов. В качестве
		элементарной операции возьмем вычисление меры близости.

	\item
		Достаточным условием, при выполнении которого
		объектно-ориентирванные модели гарантированно формируют
		качественное решение задачи топн, является выполнение транзитивности
		отношения близости на объединении объектов
		тестового, обучающего и результирующего множеств.

		Достаточным условием, при выполнении которого
		субъектно-ориентированные модели гарантированно формируют
		качественное решение задачи прогнозирования является выполнение
		транзитивности отношения близости пользователей на кластере соседей.

		Выполнение достаточных условий зависит от того,
		какие функции были выбраны в качестве мер близости
		и какие их пороговые значения установлены разработчиками или
		исследователями РС. Стоит отметить, что не всегда при разработке
		реальной РС возможно подобрать эти параметры модели так,
		чтобы выполнялись достаточные условия и РС удовлетворяла
		требованиям заказчика.

	\item Реальные исходные данные обладают свойствами динамики и
		гетерогенности. Свойство динамики заключается в том, что предпочтения
		пользователя меняются со временем. А свойство гетерогенности
		заключается в том, что пользователь предпочитает различные объекты, не
		обязательно такие, между которыми выполняется отношение близости.

		Если данные обладают свойством динамики, то эвристическое утверждение
		субъектно-лриентированных моделей может не выполняться.
		Если данные обладают свойством гетерогенности, то эвристическое утверждение
		объектно-ориентированных модели может не выполняться.
		Если эвристические утверждения не выполняются, то правила вывода
		ложны и формирования качественного решения при их применении не
		гарантированно. Таким образом, качество решений зависит от свойств
		исходных данных,  и поэтому в общем случае коллаборативные модели не
		эффективные по критерию стабильности.

	\item Асимптотические сложности алгоритмов решений задач
		коллаборативных моделей таковы, что, учитывая большие
		мощности множеств пользователей и объектов, коллаборативные
		модели нельзя назвать эффективными по критерию вычислительной
		сложности.

	\item Опишем нечеткую модель РС.
		Будем представлять контенты элементов системы в виде нечетких
		подмножеств множеств характесистик.
		Характеристиками пользователя могут быть не
		только объекты, как в случае с коллаборативной моделью,
		поэтому разрабатываемая модель является расширением
		коллаборативной по характеристикам пользователя.

	\item Между элементами системы зададим меру близости функцией
		обобщенного расстояния Хэмминга. Эта мера обладает
		метрическими свойствами. Будем считать, что если расстояние между
		объектами равно нулю, то тогда выполняется отношение близости объектов.

	\item Современные РС оперирует с большим числом различной информации о
		пользователях и объектах, за счет чего можно реализовать нечеткое
		отображение контентов пользователей на
		множество контентов объектов. К примеру, такой информацией могут
		социально-демографические данные банковской системы. В существующих
		системах подобная информация называется контекстной.
		Для того, чтоы оперивароть с контекстной информацией,
		введем нечеткое отображения на декартовом произведении
		характетистик элементов данных. Будем называть его оценкой близости
		характеристик. Оценка близости характеристик может быть задана в
		аналитической или табличной форме с
		помощью эксперта, методов машинного обучения, классификации и т.д. Введя
		оценку близости характеристик,
		можно задать отображение пользователей на множество объектов.
		Задав отображение пользователя на множество объектов, определим
		расстояние между пользователем и объектом. Значение этого расстояния
		является значением прогнозной функции, таким образом правило вывода
		нечеткой модели формально определяется через задание оценки близости
		характеристик, отображение и расстояние между пользователем и объектом.

	\item Применим введенное правило вывода нечеткой модели для решения задач.
		Задача топн решается линейным поиском близких объектов, решение задачи
		прогнозирования заключается в расчете расстояния. Асимптотическая
		сложность алгоритма решения задачи топн линейна и зависит от мощности
		множества объектов. Сложность при решении задачи прогнозирования
		константна.

		Нечеткая модель --- это тройка, в которой контенты представляются в
		виде нечетких множеств и используются коллаборватиные правила и правило
		вывода нечеткой модели.

	\item при применении правила вывода нечеткой модели качество зависит от
		рзработчиков, как и в коллаборативной модели. Однако при работе с
		коллаборативной моделью разработчик может варьировать только функцию,
		используемую в качестве меры близости и ее пороговое значение. При
		работе с нечеткой моделью разработчик обладает большими средствами,
		чтобы достичь формирование качественного решения: можно использовать
		различные данне, агоритмы, знания о предметной области и тому подобное,
		чтобы аккуратно задать правило вывода нечеткой модели.
		Правила вывода коллаборативной модели при их применении
		в нечеткой модели сегда истинны, так как выполняются достаточные
		условия формирования качественного решения, и поэтому нечеткая модел
		является эффективным расширением коллаборативной по критерию качества.
		Алгоритмы решений при применении правила пиф имеют
		асимптотическую сложность на порядок меньшую по сравнению с алгоритмами
		коллаборватиных моделей и поэтому нечеткая модель является эффективным
		расширением коллаборативной по критерию вычислительной сложности.
		Решение при применении правила пиф не зависит
		от свойств исходных данных, и поэтому нечеткая модель является
		эффективным расширением коллаборативной по критерию стабильности.

	\item Для проведение тестирование было разработано программное обеспечение,
		реализующее алгоритмы решений нечеткой, объектно-ориентированной и
		субъектно-ориентированной моделей.
		Тестирование проводились на множестве данных муви ленс.
		Эти исходные данные были собраны компанией муви ленс, которя
		занимается разработками в области рекомендательных систем. Исходные
		данные являются реальными данными, которые были заполнены реальными
		пользователями. В множестве данных участвует 670 пользователей, 10000
		объектов, являющихся фильмами и в нем около 100 000 значений
		функци ро, которые поставили сами пользователи.
		Характеристиками объектов являются кинематографиеские жанры. Число
		которых равно 18. Для нечеткой модели была сформирована
		оценка близости характеристик. Она основывается на эвристическом
		предположении о том, что
		между оценкой поставленной пользователем
		и характеристикой объекта существует корреляция. Выдвинутое
		предположение не является верным для всех пользователей, поэтому
		правило вывода нечеткой модели задано аккуратно не для каждого
		пользователя, что отразится на полученных результатах.

		Результаты
		тестирования будут представляться в виде графиков и таблиц. На оси икс
		будут располагаться идентификаторы пользователей, а по оси у ---
		значения функций, используемых в качестве оценок качества. Оранжевый
		график соотвествует нечеткой модели, голуюой --- коллаборативной. В
		таблицах будут представлены средние по всем пользователям значения
		оценок качества. Для оценки задачи топн
		топн использовалась точность, для оценки прогнозирования
		--- нормированная абсолютная погрешнось.

	\item Результаты решения задачи топн, приведенные на слайде подтверждают вывод о
		том, что применени коллаборативного правила вывода в нечеткой модели
		дает большую эффективность по критерию качества.

	\item результаты решения задачи топн, приведенные на слайде демонстрируют
		эффективность нечеткой модели при применении нечеткого правила вывода.
		качество решения при применении правила вывода нечеткой модели ниже,
		чем при применении коллаборативного правила вывода в нечеткой модели,
		так как правило вывода задано аккуратно не для каждого пользователя.

	\item Результаты прогнозирования, приведенные на слайде подтверждают вывод о
		том, что применение коллаборативного правила вывода в нечеткой модели
		дает большую эффективность по критерию качества.

	\item результаты решения задачи топн, приведенные на слайде демонстрируют
		эффективность нечеткой модели при применении нечеткого правила вывода.
		качество решения при применении правила вывода нечеткой модели ниже,
		чем при применении коллаборативного правила вывода в нечеткой модели,
		так как правило вывода задано аккуратно не для каждого пользователя.

	\item В ходе диссертационного исследовния была разработана
		математическая модель рекомендательной системы, которая
		является эффективным расширением коллабораивной и определены
		в этой модели эффективные методы решений что подтверждается
		практическими результатами. Рекомендательная система, на основании
		разработанной может быть применена в различных сферах, где
		ведется работа с пользователем, объектами некоторой предметной области
		и устанавливается взаимосвязь между ними. Это могут быть такие
		стандартные для РС области, как интернет-магазины, масс-медиа сервисы,
		так и банковские системы, определение целевых групп пользователей,
		цитирование и даже блокчейн.
\end{enumerate}
}
\end{document}
